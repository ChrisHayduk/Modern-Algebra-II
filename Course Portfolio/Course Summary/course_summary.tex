\documentclass[11pt, reqno]{amsart}
\usepackage[margin=1in]{geometry}    
\geometry{letterpaper}       
%\geometry{landscape}                % Activate for for rotated page geometry
\usepackage[parfill]{parskip}    % Deactivate to begin paragraphs with an indent rather than an empty line
\usepackage{amsfonts, amscd, amssymb, amsthm, amsmath}
\usepackage{pdfsync}  %leaves makers for tex searching
\usepackage{enumerate}
\usepackage{multicol}
\usepackage[pdftex,bookmarks]{hyperref}
\usepackage{enumitem}


\setlength\parindent{0pt}

%%% Theorems %%%--------------------------------------------------------- 
\theoremstyle{plain}
	\newtheorem{thm}{Theorem}[section]
	\newtheorem{lemma}[thm]{Lemma}
	\newtheorem{prop}[thm]{Proposition}
	\newtheorem{cor}[thm]{Corollary}
\theoremstyle{definition}
	\newtheorem*{defn}{Definition}
	\newtheorem{remark}{Remark}
\theoremstyle{example}
	\newtheorem*{example}{Example}


%%% Environments %%%--------------------------------------------------------- 
\newenvironment{ans}{\color{black}\medskip \paragraph*{\emph{Answer}.}}{\hfill \break  $~\!\!$ \dotfill \medskip }
\newenvironment{sketch}{\medskip \paragraph*{\emph{Proof sketch}.}}{ \medskip }
\newenvironment{summary}{\medskip \paragraph*{\emph{Summary}.}}{  \hfill \break  \rule{1.5cm}{0.4pt} \medskip }
\newcommand\Ans[1]{\color{black}\hfill \emph{Answer:} {#1}}


%%% Pictures %%%--------------------------------------------------------- 
%%% If you need to draw pictures, tikzpicture is one good option. Here are some basic things I always use:
\usepackage{tikz}
\usetikzlibrary{arrows}
\tikzstyle{V}=[draw, fill =black, circle, inner sep=0pt, minimum size=2pt]
\newcommand\TikZ[1]{\begin{matrix}\begin{tikzpicture}#1\end{tikzpicture}\end{matrix}}



%%% Color  %%%---------------------------------------------------------
\usepackage{color}
\newcommand{\blue}[1]{{\color{blue}#1}}
\newcommand{\NOTE}[1]{{\color{blue}#1}}
\newcommand{\MOVED}[1]{{\color{gray}#1}}


%%% Alphabets %%%---------------------------------------------------------
%%% Some shortcuts for my commonly used special alphabets and characters.
\def\cA{\mathcal{A}}\def\cB{\mathcal{B}}\def\cC{\mathcal{C}}\def\cD{\mathcal{D}}\def\cE{\mathcal{E}}\def\cF{\mathcal{F}}\def\cG{\mathcal{G}}\def\cH{\mathcal{H}}\def\cI{\mathcal{I}}\def\cJ{\mathcal{J}}\def\cK{\mathcal{K}}\def\cL{\mathcal{L}}\def\cM{\mathcal{M}}\def\cN{\mathcal{N}}\def\cO{\mathcal{O}}\def\cP{\mathcal{P}}\def\cQ{\mathcal{Q}}\def\cR{\mathcal{R}}\def\cS{\mathcal{S}}\def\cT{\mathcal{T}}\def\cU{\mathcal{U}}\def\cV{\mathcal{V}}\def\cW{\mathcal{W}}\def\cX{\mathcal{X}}\def\cY{\mathcal{Y}}\def\cZ{\mathcal{Z}}

\def\AA{\mathbb{A}} \def\BB{\mathbb{B}} \def\CC{\mathbb{C}} \def\DD{\mathbb{D}} \def\EE{\mathbb{E}} \def\FF{\mathbb{F}} \def\GG{\mathbb{G}} \def\HH{\mathbb{H}} \def\II{\mathbb{I}} \def\JJ{\mathbb{J}} \def\KK{\mathbb{K}} \def\LL{\mathbb{L}} \def\MM{\mathbb{M}} \def\NN{\mathbb{N}} \def\OO{\mathbb{O}} \def\PP{\mathbb{P}} \def\QQ{\mathbb{Q}} \def\RR{\mathbb{R}} \def\SS{\mathbb{S}} \def\TT{\mathbb{T}} \def\UU{\mathbb{U}} \def\VV{\mathbb{V}} \def\WW{\mathbb{W}} \def\XX{\mathbb{X}} \def\YY{\mathbb{Y}} \def\ZZ{\mathbb{Z}}  

\def\fa{\mathfrak{a}} \def\fb{\mathfrak{b}} \def\fc{\mathfrak{c}} \def\fd{\mathfrak{d}} \def\fe{\mathfrak{e}} \def\ff{\mathfrak{f}} \def\fg{\mathfrak{g}} \def\fh{\mathfrak{h}} \def\fj{\mathfrak{j}} \def\fk{\mathfrak{k}} \def\fl{\mathfrak{l}} \def\fm{\mathfrak{m}} \def\fn{\mathfrak{n}} \def\fo{\mathfrak{o}} \def\fp{\mathfrak{p}} \def\fq{\mathfrak{q}} \def\fr{\mathfrak{r}} \def\fs{\mathfrak{s}} \def\ft{\mathfrak{t}} \def\fu{\mathfrak{u}} \def\fv{\mathfrak{v}} \def\fw{\mathfrak{w}} \def\fx{\mathfrak{x}} \def\fy{\mathfrak{y}} \def\fz{\mathfrak{z}}
\def\fgl{\mathfrak{gl}}  \def\fsl{\mathfrak{sl}}  \def\fso{\mathfrak{so}}  \def\fsp{\mathfrak{sp}}  
\def\GL{\mathrm{GL}} \def\SL{\mathrm{SL}}  \def\SP{\mathrm{SL}}

\def\<{\langle} \def\>{\rangle}
\usepackage{mathabx}
\def\acts{\lefttorightarrow}
\def\ad{\mathrm{ad}} 
\def\Aut{\mathrm{Aut}}
\def\Ann{\mathrm{Ann}}
\def\dim{\mathrm{dim}} 
\def\End{\mathrm{End}} 
\def\ev{\mathrm{ev}} 
\def\Fr{\mathcal{F}\mathrm{r}}
\def\half{\hbox{$\frac12$}}
\def\Hom{\mathrm{Hom}} 
\def\id{\mathrm{id}} 
\def\sgn{\mathrm{sgn}}  
\def\supp{\mathrm{supp}}  
\def\Tor{\mathrm{Tor}}
\def\tr{\mathrm{tr}} 
\def\vep{\varepsilon}
\def\f{\varphi}


\def\Obj{\mathrm{Obj}}
\def\normeq{\unlhd}
\def\Set{{\cS\mathrm{et}}}
\def\Fin{{\cF\mathrm{inSet}}}
\def\Set{{\cS\mathrm{et}}}
\def\Grp{{\cG\mathrm{rp}}}
\def\Ab{{\cA\mathrm{b}}}
\def\Mod{{\cM\mathrm{od}}}
\def\ab{\mathrm{ab}}
\def\lcm{\mathrm{lcm}}
\def\ZZn{\ZZ/n\ZZ}


%%%%%%%%%%%%%%%%%%%%%%%%%%%%%% 
%%%%%%%%%%%%%%%%%%%%%%%%%%%%%%

\def\HW{5}
\def\DUE{05/25/2021}

\title[Homework \HW]{Homework \HW \\
Math A4900/44900\\
\small Due: \DUE}
\author{}

\begin{document}
%\maketitle %%% COMMENT THIS LINE OUT (add a % to the beginning of the line) and UNCOMMENT the following (delete the % symbols) to give yourself a good assignment header:
\begin{flushright}
Chris Hayduk\\
Math B4900\\
Course Summary\\
\DUE
\end{flushright}

\section{Introduction}

Building upon last semester's topics, we begin this semester by extending many of the same notions of group actions to ring theory. Modules allow us to study the behavior of rings acting on an abelian group, similarly to groups acting on sets. Modules provide us with a generalization of the notion of vector spaces, which we can recover by restricting ourselves to only considering fields acting on abelian groups. This gives us a nice link between the more abstract topics of module theory and the more foundational topics which we have studied in linear algebra. In fact, we discuss several of these topics, such as determinants and trace of matrices, while using the more advanced machinery developed throughout our study of abstract algebra. Lastly, moving into Lang's \textit{Algebra}, we make many of the ideas surrounding modules even more rigorous. In addition, we introduce the idea of exact sequences and short exact sequences, which provide us with theorems to discuss relations between modules.

\newpage

\section{Topics}

\subsection{Introduction to Module Theory}

\subsubsection{Basic Definitions and Examples}

Modules allow us to extend our notion of group actions to rings. We can think of modules as the algebraic objects which rings act on, which can be seen in the following definition \cite[\S 10.1, p. 337]{dummit}:

\par
\textbf{Definition.} Let $R$ be a ring (not necessarily commutative nor with $1$). A left $R$-module or a left module over $R$ is a set $M$ together with
\begin{enumerate}
\item a binary operation $+$ on $M$ under which $M$ is an abelian group, and
\item an action of $R$ on $M$ (that is a map $R \times M \to M$) denoted by $rm$, for all $r \in R$ and for all $m \in M$ which satisfies
\begin{enumerate}[label=\alph*)]
\item $(r+s)m = rm + sm$ for all $r, s \in R, m \in M$
\item $(rs)m = r(sm)$ for all $r,s \in R, m \in M$
\item $r(m + n) = rm + rn$ for all $r \in R, m, n \in M$\\

If the ring $R$ has a $1$ we impose the additional axiom
\item $1m = m$ for all $m \in M$
\end{enumerate}
\end{enumerate}

Now that we have our definition of modules, it is natural to consider an analogy to ``subsets'' in the context of modules \cite[\S 10.1, p. 337]{dummit}:

\par
\textbf{Definition.} Let $R$ be a ring and let $M$ be an $R$-module. An $R$-submodule of $M$ is a subgroup $N$ of $M$ which is closed under the action of ring elements, i.e., $rn \in N$, for all $r \in R, n \in N$

Hence, a submodule of $M$ is a subset of $M$ which is itself a module under the operations. Similar to the subgroup criterion, we can condense this definition for submodules into two easily verifiable conditions \cite[\S 10.1, p. 342]{dummit}:

\par
\textbf{Proposition.} Let $R$ be a ring and let $M$ be an $R$-module. A subset $N$ of $M$ is a submodule of $M$ if and only if
\begin{enumerate}
\item $N \neq \emptyset$
\item $x + ry \in N$ for all $r \in R$ and for all $x, y \in N$
\end{enumerate}

\subsubsection{Quotient Modules and Module Homomorphisms}

Now that we have discussed modules and submodules, we can also extend the notion of quotient groups and quotient rings to modules. We begin with the following definition \cite[\S 10.2, p. 345]{dummit}:

\par
\textbf{Definition.} Let $R$ be a ring and let $M$ and $N$ be $R$-modules.
\begin{enumerate}
\item A map $\varphi: M \to N$ is an $R$-module homomorphism if its respects the $R$-module structures of $M$ and $N$, i.e.
\begin{enumerate}[label=\alph*)]
\item $\varphi(x + y) = \varphi(x) + \varphi(y)$, for all $x, y \in M$ and
\item $\varphi(rx) = r\varphi(x)$ for all $r \in R, x \in M$
\end{enumerate}

\item An $R$-module homomorphism is an isomorphism (of $R$-modules) if it is both injective and surjective. The modules $M$ and $N$ are said to be isomorphic, denoted $M \cong N$, if there is some $R$-module isomorphism $\varphi: M \to N$
\item If $\varphi: M \to N$ is an $R$-module homomorphism, let ker $\varphi = \{m \in M \ | \ \varphi(m) = 0\}$ (the kernel of $\varphi$) and let $\varphi(M) = \{n \in N \ | \ n = \varphi(m) \text{ for some } m \in M\}$ (the image of $\varphi$, as usual)

\item Let $M$ and $N$ be $R$-modules and define $\Hom_R(M,N)$ to be the set of all $R$-module homomorphisms from $M$ to $N$
\end{enumerate}

The above definition allows us to extend our vocabulary regarding homomorphisms and isomorphisms from groups and rings to modules as well. This will be an important building block for constructing quotient modules, as we recall from last semester that quotient groups had a very strong connection with the kernels of homomorphisms between groups.

\par
Now similarly, to the submodule criterion, let us state a proposition that makes it much easier to verify if a map is an $R$-module homomorphism \cite[\S 10.2, p. 346]{dummit}:

\par
\textbf{Proposition.} Let $M, N$ and $L$ be $R$-modules
\begin{enumerate}
\item A map $\varphi: M \to N$ is an $R$-module homomorphism if and only if $\varphi(rx + y) =r\varphi(x) + \varphi(y)$ for all $x, y \in M$ and all $r \in R$
\item Let $\varphi, \psi$ be elements of $\Hom_R(M,N)$. Define $\varphi + \psi$ by
\begin{align*}
(\varphi + \psi)(m) = \varphi(m) + \psi(m)
\end{align*}
for all $m \in M$. Then $\varphi + \psi \in \Hom_R(M, N)$ and with this operation $\Hom_R(M, N)$ is an abelian group. If $R$ is a commutative ring than for $r \in R$ define $r\varphi$ by 
\begin{align*}
(r\varphi)(m) = r(\varphi(m))
\end{align*}

for all $m \in M$. Then $r\varphi \in \Hom_R(M, N)$ and with this action of the commutative ring $R$ the abelian group $Hom_R(M, N)$ is an $R$-module
\item If $\varphi \in \Hom_R(L, M)$ and $\psi \in \Hom_R(M, N)$ then $\psi \circ \varphi \in Hom_R(L, N)$
\item With addition as above and multiplication defined as function composition, $\Hom_R(M, M)$ is a ring with $1$. When $R$ is commutative $\Hom_R(M, M)$ is an $R$-algebra.
\end{enumerate}

Now with the definition and machinery in place to work the $R$-module homomorphisms, we can discuss quotient modules \cite[\S 10.2, p. 348]{dummit}:

\par
\textbf{Proposition.} Let $R$ be a ring, let $M$ be an $R$-module and let $N$ be a submodule of $M$. The (additive, abelian) quotient group $M/N$ can be made into an $R$-module by defining an action of elements of $R$ by
\begin{align*}
r(x+N) = (rx) + N \; \; \text{for all } r \in R, x+N \in M/N
\end{align*}

The natural projection map $\pi: M \to M/N$ defined by $\pi(x) = x + N$ is an $R$-module homomorphism with kernel $N$.

\par
Now we will also present equivalents of the isomorphism theorems for modules \cite[\S 10.2, p. 345]{dummit}:

\par
\textbf{Theorem.} Isomorphism Theorems
\begin{enumerate}
\item (\textit{The First Isomorphism Theorem for Modules}) Let $M, N$ be $R$-modules and let $\varphi: M \to N$ be an $R$-module homomorphism. Then ker$\varphi$ is a submodule of $M$ and $M/\text{ker}\varphi \cong \varphi(M)$
\item (\textit{The Second Isomorphism Theorem}) Let $A, B$ be submodules of the $R$-module $M$. Then $(A+B)/B \cong A/(A \cap B)$
\item (\textit{The Third Isomorphism Theorem}) Let $M$ be an $R$-module, and let $A$ and $B$ be submodules of $M$ with $A \subset B$. Then $(M/A)/(B/A) \cong M/B$
\item (\textit{The Fourth Isomorphism Theorem}) Let $N$ be a submodule of the $R$-module $M$. There is a bijection between the submodules of $M$ which contain $N$ and the submodules of $M/N$. The correspondence is given by $A \longleftrightarrow A/N$, for all $A \supset N$. This correspondence commutes with the process of taking sums and intersections (i.e., there is a lattice isomorphism between the lattice of submodules of $M/N$ and the lattice of submodules of $M$ which contain $N$).
\end{enumerate}

\subsubsection{Generation of Modules and Direct Sums}

Now we introduce some definitions regarding submodules generated by subsets as well as finite sums of submodules \cite[\S 10.3, p. 351]{dummit}:

\par
\textbf{Definition.} Let $M$ be an $R$-module and let $N_1, \ldots, N_n$ be submodules of $M$.
\begin{enumerate}
\item The \textit{sum} of $N_1, \ldots, N_n$ is the set of all finite sums of elements from the sets $N_i: \{a_1 + a_2 + \cdots + a_n \  | \ a_i \in N_i \text{ for all } i\}$. Denote this sum by $N_1 + \cdots + N_n$
\item For any subset $A$ of $M$ let
\begin{align*}
RA = \{r_1 a_1 + r_2a_2 + \cdots + r_ma_m \ | \ r_1, \ldots, r_m \in R, a_1, \ldots, a_m \in A, m \in \mathbb{Z}^+\}
\end{align*}

If $A$ is the finite set $\{a_1, a_2, \ldots, a_n\}$ we shall write $Ra_1 + Ra_2 + \cdots + Ra_n$ for $RA$. Call $RA$ the \textit{submodule of M generated by A}. If $N$ is a submodule of $M$ (possibly $N = M$) and $N = RA$, for some subset $A$ of $M$, we call $A$ a \textit{set of generators} or \textit{generating set} for $N$, and we say $N$ is \textit{generated by} $A$

\item A submodule $N$ of $M$ (possibly $N = M$) is \textit{finitely generated} if there is some finite subset $A$ of $M$ such that $N = RA$, that is, if $N$ is generated by some finite subset

\item A submodule $N$ of $M$ (possibly $N = M$) is \textit{cyclic} if there exists an element $a \in M$ such that $N = Ra$, that is, if $N$ is generated by one element:

\begin{align*}
N = Ra = \{ra \ | \ r \in R\}
\end{align*}
\end{enumerate} 

The direct product of a collection of $R$-modules is again an $R$-module. We can also refer to this direct product as an \textit{(external) direct sum} \cite[\S 10.3, p. 353]{dummit}. We can characterize some equivalent notions of direct products of $R$-modules as follows \cite[\S 10.3, p. 353]{dummit}:

\par
\textbf{Proposition 5.} Let $N_1, N_2, \ldots, N_k$ be submodules of the $R$-module $M$. Then the following are equivalent:
\begin{enumerate}
\item The map $\pi: N_1 \times N_2 \times \cdots \times N_k \to N_1 + N_2 + \cdots N_k$ defined by $$\pi(a_1, a_2, \ldots, a_k) = a_1 + a_2 + \ldots + a_k$$ is an isomorphism (of $R$-modules): $N_1 + N_2 + \cdots + N_k \cong N_1 \times N_2 \times \cdots N_k$
\item $N_j \cap (N_1 + N_2 + \cdots + N_{j-1} + N_{j_1} + \cdots + N_k) = 0$ for all $j \in \{1, 2, \ldots, k\}$
\item Every $x \in N_1 + \cdots + N_k$ can be written \textit{uniquely} in the form $a_1 + a_2 + \cdots + a_k$ with $a_i \in N_i$.
\end{enumerate}

If an $R$-module $M = N_1 + N_2 + \cdots + N_k$ is the sum of submodules $N_1, N_2, \ldots, N_k$ of $M$ satisfying the equivalent conditions of the proposition above, then $M$ is said to be the \textit{(internal) direct sum} of $N_1, N_2, \ldots, N_k$, written $$M = N_1 \oplus N_2 \oplus \cdots \oplus N_k$$

We also have an additional way to show that an $A$-module is decomposible into a direct sum of $A$-module through the following proposition \cite[Lec 11, p. 2]{dau}:

\par
\textbf{Proposition:} Let $M$ be an $A$-module. Suppose that there exist $A$-module homomorphisms $\varphi_i: M \to M$ for $i = 1, \ldots, n$ such that $$\sum_{i=1}^n \varphi_i = \id \; \text{ and } \; \varphi_i \circ \varphi_j = 0 \text{ for } i \neq j$$ Then $\varphi_i^2 = \varphi_i$ for all $i$. Further, if $X_i = \varphi_i(M)$, then $$\varphi: M \to \bigoplus_{i=1}^n X_i \; \text{ defined by } \varphi = \varphi_1 \oplus \cdot \varphi_n$$ is an $A$-module 

\subsubsection{Decomposition of $\CC S_3$}

Let $A = \CC S_3$. Then if we define, 
\begin{align*}
&z_1 = \frac{1}{6}(1 + (12) + (13) + (23) + (123) + (132)),\\
&z_2 = \frac{1}{6}(1 - (12) - (13) - (23) + (123) + (132)),\\
&\text{and } z_3 = \frac{1}{3}(2 - (123) - (132)),
\end{align*}

we showed that the maps,
\begin{align*}
\varphi_1: M \to M, m \mapsto z_1m, \; \; \varphi_2: M \to M, m \mapsto z_2m, \; \; \varphi_3: M \to M, m \mapsto z_3m
\end{align*}

satisfy,
\begin{align*}
\varphi_1 + \varphi_2 + \varphi_3 = \id_M \; \; \text{ and } \; \; \varphi_i \varphi_j = 0 \text{ for all } i \neq j
\end{align*}

By the proposition presented above, we thus have that for any $A$-module $M$, we have $$M \cong \varphi_1(M) \oplus \varphi_2(M) \oplus \varphi_3(M)$$

\subsubsection{Free Modules}

\par
We can now introduce the definition for a free module in order to expand on this notion \cite[\S 10.3, p. 354]{dummit}:

\par
\textbf{Definition.} An $R$-module $F$ is said to be \textit{free} on the subset $A$ of $F$ if for every nonzero element $x$ of $F$, there exist unique nonzero elements $r_1, r_2, \ldots, r_n$ of $R$ and unique $a_1, a_2, \ldots, a_n$ in $A$ such that $x = r_1a_1 + r_2a_2 + \cdots + r_na_n$, for some $n \in \ZZ^+$. In this situation we say $A$ is a \textit{basis} or \textit{set of free generators} for $F$. If $R$ is a commutative ring, the cardinality of $A$ is called the \textit{rank} of $F$.

These free generators satisfy an interesting property \cite[\S 10.3, p. 354]{dummit}:

\par
\textbf{Theorem.} For any set $A$ there is a free $R$-module $F(A)$ on the set $A$ and $F(A)$ satisfies the following \textit{universal property}: if $M$ is any $R$-module and $\varphi: A \to M$ is any map of sets, then there is a unique $R$-module homomorphism $\Phi: F(A) \to M$ such that $\Phi(A) = \varphi(a)$ for all $a \in A$.

\par
When $A$ is the finite set $\{a_1, a_2, \ldots, a_n\}$, $F(A) = Ra_1 \oplus Ra_2 \oplus \cdots \oplus Ra_n \cong R^n$.


%%%%%%%%%%%%%%%%%%%%%%%%%%%%%%%%%%%%%%%%%%%%%%%%%%%%%%%%%%%%%%%%%%%%%%%%%%%%%%%%%%%%%%%%
\newpage
\subsection{Vector Spaces}


\subsubsection{Definitions and Basic Theory}

Vector spaces follow very nicely from our above discussion of modules, as the definition of each vector space property is the same as the module-theoretic definition with the added assumption that $R$ is a field. Hence, in the definitions above, we can replace ``module'' with ``vector space'' to get the equivalent definitions. Now we will discuss some notions distinct from modules \cite[\S 11.1, p. 409]{dummit}:

\par
\textbf{Definition.} \begin{enumerate}
\item A subset $S$ of $V$ is called a set of \textit{linearly independent} vectors if an equation $\alpha_1 v_1 + \alpha_2 v_2 + \cdots + \alpha_n v_n = 0$ with $\alpha_1, \alpha_2, \ldots, \alpha_n \in F$ and $v_1, v_2, \ldots, v_n \in S$ implies $\alpha_1 = \alpha_2 = \cdots = \alpha_n = 0$
\item A \textit{basis} of a vector space $V$ is an ordered set of linearly independent vectors which span $V$. In particular two bases will be considered different even if one is simply a rearrangement of the other. This is sometimes referred to as an \textit{ordered basis}.
\end{enumerate}

\textbf{Proposition.} Assume the set $\mathcal{A} = \{v_1, v_2, \ldots, v_n\}$ spans the vector space $V$ but no proper subset of $\mathcal{A}$ spans $V$. Then $\mathcal{A}$ is a basis of $V$. In particular, any finitely generated (i.e., finitely spanned) vector space over $F$ is a free $F$-module

\par
\textbf{Corollary.} Assume the finite set $\mathcal{A}$ spans the vector space $V$. Then $\mathcal{A}$ contains a basis of $V$.

\subsubsection{The Matrix of a Linear Transformation}

We have that $M_{\mathcal{B}}^{\mathcal{E}}(\varphi) = (a_{ij})$ is the $m \times n$ matrix whose $i, j$ entry is $\alpha_{ij}$. Then for $v \in V$, we can write $v$ in terms of the basis $\mathcal{B}$ as follows $$v = \sum_{i=1}^n \alpha_i v_i, \ \alpha_i \in F$$Then the image of $v$ under $\varphi$ is given by $$\varphi(v) = \sum_{i=1}^m \beta_i w_i$$Thus we have the following definition \cite[\S 11.2, p. 415]{dummit}:

\par
\textbf{Definition.} The $m \times n$ matrix $A = (a_{ij})$ associated to the linear transformation $\varphi$ above is said to \textit{represent} the linear transformation $\varphi$ with respect to the bases $\mathcal{B}, \mathcal{E}$. Similarly $\varphi$ is the linear transformation represented by $A$ with respect to the bases $\mathcal{B}, \mathcal{E}$.


\subsubsection{Dual Vector Spaces}

We will begin with the definition of dual spaces \cite[\S 11.3, p. 431]{dummit}:

\par
\textbf{Definition.}
\begin{enumerate}
\item For $V$ any vector space over $F$ let $V^* = \Hom_F(V,F)$ be the space of linear transformation from $V$ to $F$, called the \textit{dual space} of $V$. Elements of $V^*$ are called \textit{linear functionals}.
\item If $\mathcal{B} = \{v_1, v_2, \ldots, v_n\}$ is a basis of the finite dimensional space $V$, define $v_1^8 \in V^*$ for each $i \in \{1, 2, \ldots, n\}$ by its action on the basis $\mathcal{B}$:
\begin{align*}
v_i^*(v_j) = \begin{cases} 
      1, \text{ if } i = j \\
      0, \text{ if } i \neq j 
   \end{cases}
\end{align*}
\end{enumerate}

We can now use the notation we've just introduced in order to discuss some properties of $V^*$ \cite[\S 11.3, p. 432]{dummit}:

\par
\textbf{Proposition.} With notation as above, $\{v_1^*, v_2^*, \ldots, v_n^*\}$ is a basis of $V^*$. In particular, if $V$ is finite dimensional then $V^*$ has the same dimension as $V$.

\par
\textbf{Definition.} The basis $\{v_1^*, v_2^*, \ldots, v_n^*\}$ of $V^*$ is called the \textit{dual basis} to $\{v_1, v_2, \ldots, v_n\}$

\par
\textbf{Definition.} The dual of $V^*$, namely $V^{**}$, is called the \textit{double dual} or \textit{second dual} of $V$.

The dual space to $V^*$, $V^{**}$, can be related back to $V$ through a linear transformation which is reflected in the following theorem \cite[\S 11.3, p. 433]{dummit}:

\par
\textbf{Theorem.} There is a natural injective linear transformation from $V$ to $V^{**}$. If $V$ is finite dimensional then this linear transformation is an isomorphism.

\par
Hence, we can relate $V$ both to its dual space and the dual of its dual space.

\subsubsection{Determinants}

We can start with the definition of what a determinant is, as well as how to compute it,

\par
\textbf{Definition.} An $n \times n$ \textit{determinant function} on $R$ is any function $$\text{det}: M_{n \times n}(R) \to R$$that satisfies the following two axioms:
\begin{enumerate}
\item det is an $n$-multilinear alternating form on $R^n (= V)$, where the $n$-tuples are the $n$ columns of the matrices in $M_{n \times n}(R)$
\item $\text{det}(I) = 1$ where $I$ is the $n \times n$ identity matrix
\end{enumerate}

\par
\textbf{Theorem.} There is a unique $n \times n$ determinant function on $R$ and it can be computed for any $n \times n$ matrix $(\alpha_{ij})$ by the formula:
\begin{align*}
\text{det}(\alpha_{ij}) = \sum_{\sigma \in S_n} \epsilon(\sigma) \alpha_{\sigma(1)1}\alpha_{\sigma(2)2}\cdots \alpha_{\sigma(n)n}
\end{align*}

%%%%%%%%%%%%%%%%%%%%%%%%%%%%%%%%%%%%%%%%%%%%%%%%%%%%%%%%%%%%%%%%%%%%%%%%%%%%%%%%%%%%%%%%%


\subsection{Modules over Principal Ideal Domains}

\subsubsection{Jordan Canonical Form}

Jordan Canonical Form allows us to ``simplify'' matrices, putting them into a form that is as close to a diagonal matrix as possible \cite[\S 11.3, p. 431]{dummit}:

\par
\textbf{Definition.}
\begin{enumerate}
\item A matrix is said to be in \textit{Jordan canonical form} if it is a block diagonal matrix with Jordan blocks along the diagonal
\item A \textit{Jordan canonical form} for a linear transformation $T$ is a matrix representing $T$ which is in Jordan canonical form
\end{enumerate}

That is, a matrix $J$ is in Jordan canonical form if it is constructed as follows,
\begin{align*}
J = \begin{bmatrix}
J_1 & \;     & \; \\
\;  & \ddots & \; \\ 
\;  & \;     & J_p\end{bmatrix}
\end{align*}

where each block $J_i$ is a square matrix of the form,
\begin{align*}
J_i = \begin{bmatrix}
\lambda_i & 1            & \;     & \;  \\
\;        & \lambda_i    & \ddots & \;  \\
\;        & \;           & \ddots & 1   \\
\;        & \;           & \;     & \lambda_i       
\end{bmatrix}
\end{align*}

\par
\textbf{Theorem.} Let $V$ be a finite dimensional vector space over the field $F$ and let $T$ be a linear transformation of $V$. Assume $F$ contains all the eigenvalues of $T$:
\begin{enumerate}
\item There is a basis for $V$ with respect to which the matrix for $T$ is in Jordan canonical form, i.e., is a block diagonal matrix whose diagonal blocks are the Jordan blocks for the elementary divisors of $V$
\item The Jordan canonical form for $T$ is unique up to a permutation of the Jordan blocks along the diagonal
\end{enumerate}

%%%%%%%%%%%%%%%%%%%%%%%%%%%%%%%%%%%%%%%%%%%%%%%%%%%%%%%%%%%%%%%%%%%%%%%%%%%%%%%%%%%%%%%%%

\subsection{The Group of Homomorphisms}

We have that a pair of homomorphisms is said to be exact if $$X \xrightarrow{\varphi} Y \xrightarrow{\psi} Z$$ if $\text{img}(\varphi) = \ker(\psi)$. This definition may seem a bit esoteric, so we'll present an example here in order to illustrate things better:
\begin{align*}
\ZZ \xrightarrow{2x} \ZZ \xrightarrow{x \mod 2} \ZZ/2\ZZ
\end{align*}

We see that the map $2x$ has the image $2\ZZ$. Moreover, the kernel of $x \mod 2$ is precisely $2\ZZ$ as well because every even integer modulo 2 is equal to $0$. Hence, we have that this sequence is exact.

\par
We can now define a special case of an exact sequence \cite[\S 10.5, p.379]{dummit}:

\par
\textbf{Definition:} The exact sequence $$0 \rightarrow A \xrightarrow{\psi} B \xrightarrow{\phi} \rightarrow C \rightarrow 0$$ is called a \textit{short exact sequence}.

\par
We can extend our exact sequence from above to a short exact sequence as follows,
\begin{align*}
0 \rightarrow \ZZ \xrightarrow{2x} \ZZ \xrightarrow{x \mod 2} \ZZ/2\ZZ \rightarrow 0
\end{align*}

\par
We can now present two similar relationships between exact sequences of modules and of the associated homomorphism groups \cite[\S 2.2, p. 122]{lang}:

\par
\textbf{Proposition.} A sequence 
$$X' \xrightarrow{\lambda} X \to X'' \to 0$$ is exact if and only if the sequence $$\Hom_A(X', Y) \leftarrow \Hom_A(X, Y) \leftarrow  \Hom_A(X'', Y) \leftarrow 0$$ is exact for all $Y$.

\par
Similarly, we have,

\par
\textbf{Proposition.} A sequence $$0 \rightarrow Y' \rightarrow Y \rightarrow Y'',$$ is exact if and only if $$0 \rightarrow \Hom_A(X, Y') \rightarrow \Hom_A(X, Y) \rightarrow \Hom_A(X, Y'')$$ is exact for all $X$.

\par
In addition, we have the notion of a split short exact sequence \cite[Lec 10, p.6]{dau}:

\newpage
\par
\textbf{Definition:} An SES $0 \rightarrow X \xrightarrow{f} Y \xrightarrow{g} Z \rightarrow 0$ of $A$-modules is said to be \textit{split} if there is \textit{some} $Z' \subset Y$ satisfying,
\begin{enumerate}
\item $Y = f(X) + Z'$
\item $f(X) \cap Z' = 0$, and
\item $g|_{Z'}$ is an isomorphism.
\end{enumerate}

So $Y \cong f(X) \oplus Z'$, and that isomorphism preserves the splitting.

\par
From the same lecture, we have two examples demonstrating a difference between split and non-split short exact sequences, given by \cite[Lec 10, p.7]{dau}:,
\begin{align}
0 \rightarrow \ZZ \xrightarrow{x \mapsto (x,0)} \ZZ \oplus \ZZ/n\ZZ \xrightarrow{(x,y) \mapsto y} \ZZ/n\ZZ \rightarrow 0
\end{align}

and,

\begin{align}
0 \rightarrow \ZZ \xrightarrow{x \mapsto nz} \ZZ \xrightarrow{x \mapsto x + n\ZZ} \ZZ/n\ZZ \rightarrow 0
\end{align}

are both exact, but (1) is split and (2) is not.

%%%%%%%%%%%%%%%%%%%%%%%%%%%%%%%%%%%%%%%%%%%%%%%%%%%%%%%%%%%%%%%%%%%%%%%%%%%%%%%%%%%%%%%%%

\subsection{Continuing with Modules}

Using the above notions from exact sequences, we can now define some more general classes of modules.

\subsubsection{Projective Modules}

Projective modules generalize the notion of free modules defined in Section 2.1.5 (recall that free modules are modules with basis vectors). We will see that every free module is a projective module, but the converse fails to hold true in some cases.

\par
The following proposition fully characterizes projective modules \cite[\S 10.5, p. 389]{dummit}:

\par
\textbf{Proposition.} Let $P$ be an $R$-module. Then the following are equivalent:
\begin{enumerate}
\item For any $R$-modules $L$, $M$, and $N$, if $$0 \rightarrow L \xrightarrow{\psi} M \xrightarrow{\varphi} N \rightarrow 0$$ is a short exact sequence, then $$0 \rightarrow \Hom_R(P, L) \xrightarrow{\psi'} \Hom_R(P, M) \xrightarrow{\varphi'} \Hom_R(P, N) \rightarrow 0$$ is also a short exact sequence.

\item For any $R$-modules $M$ and $N$, if $M \xrightarrow{\varphi} N \rightarrow 0$ is exact, then every $R$-module homomorphism from $P$ into $N$ lifts to an $R$-module homomorphism into $M$

\item If $P$ is a quotient of the $R$-module $M$ then $P$ is isomorphic to a direct summand of $M$, i.e., every short exact sequence $0 \rightarrow L \rightarrow M \rightarrow P \rightarrow 0$ splits.

\item $P$ is a direct summand of a free $R$-module
\end{enumerate}

An $R$-module $P$ is called \textit{projective} if it satisfies any of the above conditions.

\subsubsection{Injective Modules}

Injective modules form the dual of projective modules. Generally, it is a module that shares some properties with the $\ZZ$ module $\QQ$ of all rational numbers. Similarly to projective modules, we will present the proposition which characterizes injective modules \cite[\S 10.5, p. 394]{dummit}:

\newpage
\textbf{Proposition.} Let $Q$ be an $R$-module. Then the following are equivalent:
\begin{enumerate}
\item For any $R$-modules $L$, $M$, and $N$, if $$0 \rightarrow L \xrightarrow{\psi} M \xrightarrow{\varphi} N \rightarrow 0$$ is a short exact sequence, then $$0 \rightarrow \Hom_R(N, Q) \xrightarrow{\varphi'} \Hom_R(M, Q) \xrightarrow{\psi'} \Hom_R(L, Q) \rightarrow 0$$ is also a short exact sequence.

\item For any $R$-modules $L$ and $M$, if $0 \rightarrow L \xrightarrow{\psi} M$ is exact, then every $R$-module homomorphism from $L$ into $Q$ lifts to an $R$-module homomorphism of $M$ into $Q$

\item If $Q$ is a submodule of the $R$-module $M$, then $Q$ is a direct summand of $M$, i.e., every short exact sequence $0 \rightarrow Q \rightarrow M \rightarrow N \rightarrow 0$ splits
\end{enumerate}

An $R$-module $Q$ is called \textit{injective} if it satisfies any of the above equivalent conditions. For example, the rational numbers $\QQ$ are an injective $\ZZ$-module. However, since $\ZZ$ is not divisible, $\ZZ$ is not an injective $\ZZ$-module.

\par
We now also have an important theorem which characterizes the containment of modules in injective modules \cite[\S 10.5, p. 398]{dummit}:

\par
\textbf{Theorem.} Let $R$ be a ring with $1$ and let $M$ be an $R$-module. Then $M$ is contained in an injective $R$-module.

\subsubsection{Schur's Lemma} Below is the statement of Schur's Lemma \cite[Lec. 15, p. 1]{dau}:

\par
\textbf{Theorem.} (Schur's Lemma) If $U$ and $V$ are simple $A$-modules and $\varphi: U \to V$ is an $A$-module homomorphism, then $\varphi$ is either the $0$ map, or it is an isomorphism.

Some consequences of this lemma are presented below:
\begin{enumerate}
\item if $V$ is simple, then $$\End_A(V) = \{A-\text{module homomorphisms } \varphi: V \to V\}$$ is a division ring, i.e. $\End_A(V) = \Aut_A(V) \cup \{0\}$

\item If $U$ is a simple submodule of an $A$-module $V$ and $\varphi: V \to W$ is an $A$-module homomorphism, then $U \subset \ker(\varphi)$ or $W$ contains a submodule isomorphic to $U$.

\item If $V \cong V_1 \oplus V_2$ where $V_1$ and $V_2$ are simple nonisomorphic modules, then the only submodules of $V$ are $0, V_1, V_2$, and $V$. Further, $$\End_A(V) = \End_A(V_1) \times \End_A(V_2)$$ and, again, $\End_A(V_i)$ are division rings.

\item If $V$ is simple and $U$ is a nonzero proper submodule of $V \oplus V$, then $V \cong U$ (but is not necessarily equal to $V \oplus 0$ or $0 \oplus V$)
\end{enumerate}

\subsubsection{Maschke's Theorem} We have that a ring $A$ is \textit{semisimple} if every $A$ module is isomorphic to the direct sum of simple modules \cite[Lec. 11, p. 1]{dau}. We will now state Maschke's Theorem, which states that ``group algebras are often semisimple'' \cite[Lec. 11, p. 2]{dau}:

\par
\textbf{Theorem.} Let $G$ be a finite group and let $F$ be a field with $\text{char}(F) \not\divides |G|$. Let $V$ be an $FG$-module and $U$ be a submodule. Then there's a submodule $W$ in $V$ satisfying $U \cap W = 0$ and $U + W = V$ so that $V \cong U \oplus W$, i.e. $V$ contains a direct sum complement to $U$.

\subsubsection{Artin-Wedderburn}

Artin-Wedderburn assists in classifying semisimple rings and semisimple algebras. The theorem statement is as follows \cite[Lec. 16, p. 6]{dau}:

\par
\textbf{Theorem.} Let $A \neq 0$ be a ring with $1$. The following are equivalent:
\begin{enumerate}
\item Every $A$-module is projective

\item Every $A$-module is injective

\item Every $A$-module is completely reducible

\item The left regular module decomposes as $A \cong \bigoplus_{\lambda \in \Lambda} A^{\lambda}$, where each $A^{\lambda}$ is a simple $A$-module. More specifically, $A^{\lambda} = Ae_{\lambda}$ for some idempotent $e_{\lambda} \in A$, and the idempotents $\{e_{\lambda} \ | \ \lambda \in \Lambda\}$ satisfy 
\begin{align*}
e_{\lambda}e_{\mu} = \delta_{\lambda_{\mu}}e_{\lambda} \; \; \text{ and } \; \; \sum_{\lambda \in \Lambda} e_{\lambda} = 1
\end{align*}

\item As rings, $$A \cong M_{n_1}(\Delta_1) \times \cdots \times M_{n_{\ell}}(\Delta_{\ell})$$ where $\Delta_i$ is a division ring for each $i$. Moreover, up to permutation and isomorphism, the $n_i$ and $\delta_i$ are unique.
\end{enumerate}

Any ring $A$ satisfying these conditions is called \textit{semisimple}.

\par
Example: If $G$ is a finite group and $F$ is a field of characteristic not dividing $|G|$, then $FG$ is semisimple \cite[Lec. 16, p. 6]{dau}.

\subsubsection{Noetherian Modules}

Noetherian modules are modules which satisfy the ascending chain condition on its submodules, defined formally as follows \cite[\S 10.1, p. 413]{lang}:

\par
\textbf{Definition.} Let $A$ be a ring and $M$ a module (i.e., a left $A$-module). We shall say that $M$ is \textit{Noetherian} if it satisfies any of the following three conditions:
\begin{enumerate}
\item Every submodule of $M$ is finitely generated

\item Every ascending sequence of submodules of $M$, $$M_1 \subset M_2 \subset M_3 \subset \cdots$$ such that $M_i \neq M_{i+1}$ is finite.

\item Every non-empty set $S$ of submodules of $M$ has a maximal element (i.e., a submodule $M_0$ such that for any element $N$ of $S$ which contains $M_0$ we have $N = M_0$).
\end{enumerate}

All three of the above conditions are equivalent. We can also characterize the submodules of a Noetherian module as follows \cite[Lec. 17, p. 4]{dau}:

\par
\textbf{Proposition.} If $M$ is Noetherian, then so is $N$ and $M/N$ for every submodule $N \subset M$.

\par
Moreover, we can go in the reverse direction, and show that a Noetherian submodule can imply that the greater module is also Noetherian \cite[Lec. 17, p. 4]{dau}:

\par
\textbf{Proposition.} For a submodule $N \subset M$, if $N$ and $M/N$ are Noetherian, then so is $M$.

\subsubsection{Artinian Modules}

Artinian modules are modules which satisfy the descending chain condition on its submodules, defined formally as follows \cite[\S 10.7, p. 439]{lang}:

\par
\textbf{Definition.} Let $A$ be a ring, not necessarily commutative, and $E$ an $A$-module. We say that $E$ is \textit{Artinian} if $E$ satisfies the descending chain condition on submodules, that is a sequence $$E_1 \supset E_2 \supset E_3 \supset \cdots$$ must stabilize: there exists an integer $N$ such that if $n \geq N$ then $E_n = E_{n+1}$.

A ring is Artinian if its left-regular module is Artinian.

\subsubsection{Semisimple Modules}

Put succinctly, a semisimple module is a module that, in some sense, are modules which can be easily understood by their constituent parts. We will define it formally as follow \cite[Lec 18, p. 1]{dau}:

\par
\textbf{Definition.} Let $M$ be an $A$-module. We say $M$ is semisimple if every submodule of $M$ is a direct summand of $M$ (i.e. every submodule of $M$ has a direct sum complement).

\par
Some examples include: the $0$ module, simple modules, and completely decomposable modules

\par
We can define some similar notions of semisimple modules through the following theorem \cite[\S 1.2, p. 26]{lam}:

\par
\textbf{Theorem.} For an $A$-module $M$, the following three properties are equivalent:
\begin{enumerate}
\item $M$ is semisimple.

\item $M$ is the direct sum of a family of simple submodules.

\item $M$ is the sum of a family of simple submodules.
\end{enumerate}

\par
Moreover, we can state a theorem which characterizes equivalent notions of \textit{semisimple rings} \cite[\S 1.2, p. 27]{lam}:

\par
\textbf{Theorem.} For a ring $A$, the following are equivalent:
\begin{enumerate}
\item Every SES of $A$-modules splits.

\item Every $A$-module is semisimple.

\item Every finitely generated $A$-module is semisimple.

\item Every cyclic $A$-module is semisimple.

\item The left-regular $A$-module is semisimple
\end{enumerate}

Moreover, if $A$ satisfies these conditions, then the left-regular module is isomorphic to a finite direct sum of simple modules.

\par
We can now connect semisimplicity to projective and injective modules as follows \cite[Lec. 18, p. 4]{dau}:

\par
\textbf{Theorem.} Let $A$ be a ring with 1. The following are equivalent:
\begin{enumerate}
\item $A$ is semisimple.
\item All $A$-modules are projective.
\item All finitely-generated $A$-modules are porjective.
\item All cyclic $A$-modules are projective.
\end{enumerate}

\par
\textbf{Corollary.} $A$ is semisimple if and only if every $A$-module is injective.

%%%%%%%%%%%%%%%%%%%%%%%%%%%%%%%%%%%%%%%%%%%%%%%%%%%%%%%%%%%%%%%%%%%%%%%%%%%%%%%%%%%%%%%%%


\subsection{Matrix Rings}

%%%%%%%%%%%%%%%%%%%%%%%%%%%%%%%%%%%%%%%%%%%%%%%%%%%%%%%%%%%%%%%%%%%%%%%%%%%%%%%%%%%%%%%%%


\subsection{Categories}

%%%%%%%%%%%%%%%%%%%%%%%%%%%%%%%%%%%%%%%%%%%%%%%%%%%%%%%%%%%%%%%%%%%%%%%%%%%%%%%%%%%%%%%%%

\subsection{Tensor Products}

%%%%%%%%%%%%%%%%%%%%%%%%%%%%%%%%%%%%%%%%%%%%%%%%%%%%%%%%%%%%%%%%%%%%%%%%%%%%%%%%%%%%%%%%%

\subsection{Lie Groups and Lie Algebras}

%%%%%%%%%%%%%%%%%%%%%%%%%%%%%%%%%%%%%%%%%%%%%%%%%%%%%%%%%%%%%%%%%%%%%%%%%%%%%%%%%%%%%%%%%

\subsection{Characters}


%%%%%%%%%%%%%%%%%%%%%%%%%%%%%%%%%%%%%%%%%%%%%%%%%%%%%%%%%%%%%%%%%%%%%%%%%%%%%%%%%%%%%%%%%
%%%%%%%%%%%%%%%%%%%%%%%%%%%%%%%%%%%%%%%%%%%%%%%%%%%%%%%%%%%%%%%%%%%%%%%%%%%%%%%%%%%%%%%%%
%%%%%%%%%%%%%%%%%%%%%%%%%%%%%%%%%%%%%%%%%%%%%%%%%%%%%%%%%%%%%%%%%%%%%%%%%%%%%%%%%%%%%%%%%




\newpage
\begin{thebibliography}{9}

\bibitem{dummit}
Dummit, David Steven., and Richard M. Foote. \textit{Abstract Algebra}. 3rd ed., John Wiley \& Sons, 2004. 

\bibitem{dau}
Daugherty, Zajj. \textit{Math B4900: Modern Algebra II}, Lecture Notes, 2021.

\bibitem{lam}
Lam, Tsit-Yuen. \textit{A First Course in Noncommutative Rings}. Springer, 2001. 

\bibitem{lang}
Lang, Serge. \textit{Algebra}. Springer-Verlag New York Inc., 2012. 

\end{thebibliography}


\end{document}