\documentclass[11pt, reqno]{amsart}
\usepackage[margin=1in]{geometry}    
\geometry{letterpaper}       
%\geometry{landscape}                % Activate for for rotated page geometry
\usepackage[parfill]{parskip}    % Deactivate to begin paragraphs with an indent rather than an empty line
\usepackage{amsfonts, amscd, amssymb, amsthm, amsmath}
\usepackage{pdfsync}  %leaves makers for tex searching
\usepackage{enumerate}
\usepackage{multicol}
\usepackage[pdftex,bookmarks]{hyperref}




%%% Theorems %%%--------------------------------------------------------- 
\theoremstyle{plain}
	\newtheorem{thm}{Theorem}[section]
	\newtheorem{lemma}[thm]{Lemma}
	\newtheorem{prop}[thm]{Proposition}
	\newtheorem{cor}[thm]{Corollary}
\theoremstyle{definition}
	\newtheorem*{defn}{Definition}
	\newtheorem{remark}{Remark}
\theoremstyle{example}
	\newtheorem*{example}{Example}


%%% Environments %%%--------------------------------------------------------- 
\newenvironment{ans}{\color{black}\medskip \paragraph*{\emph{Answer}.}}{\hfill \break  $~\!\!$ \dotfill \medskip }
\newenvironment{sketch}{\medskip \paragraph*{\emph{Proof sketch}.}}{ \medskip }
\newenvironment{summary}{\medskip \paragraph*{\emph{Summary}.}}{  \hfill \break  \rule{1.5cm}{0.4pt} \medskip }
\newcommand\Ans[1]{\color{black}\hfill \emph{Answer:} {#1}}


%%% Pictures %%%--------------------------------------------------------- 
%%% If you need to draw pictures, tikzpicture is one good option. Here are some basic things I always use:
\usepackage{tikz}
\usetikzlibrary{arrows}
\tikzstyle{V}=[draw, fill =black, circle, inner sep=0pt, minimum size=2pt]
\newcommand\TikZ[1]{\begin{matrix}\begin{tikzpicture}#1\end{tikzpicture}\end{matrix}}



%%% Color  %%%---------------------------------------------------------
\usepackage{color}
\newcommand{\blue}[1]{{\color{blue}#1}}
\newcommand{\NOTE}[1]{{\color{blue}#1}}
\newcommand{\MOVED}[1]{{\color{gray}#1}}


%%% Alphabets %%%---------------------------------------------------------
%%% Some shortcuts for my commonly used special alphabets and characters.
\def\cA{\mathcal{A}}\def\cB{\mathcal{B}}\def\cC{\mathcal{C}}\def\cD{\mathcal{D}}\def\cE{\mathcal{E}}\def\cF{\mathcal{F}}\def\cG{\mathcal{G}}\def\cH{\mathcal{H}}\def\cI{\mathcal{I}}\def\cJ{\mathcal{J}}\def\cK{\mathcal{K}}\def\cL{\mathcal{L}}\def\cM{\mathcal{M}}\def\cN{\mathcal{N}}\def\cO{\mathcal{O}}\def\cP{\mathcal{P}}\def\cQ{\mathcal{Q}}\def\cR{\mathcal{R}}\def\cS{\mathcal{S}}\def\cT{\mathcal{T}}\def\cU{\mathcal{U}}\def\cV{\mathcal{V}}\def\cW{\mathcal{W}}\def\cX{\mathcal{X}}\def\cY{\mathcal{Y}}\def\cZ{\mathcal{Z}}

\def\AA{\mathbb{A}} \def\BB{\mathbb{B}} \def\CC{\mathbb{C}} \def\DD{\mathbb{D}} \def\EE{\mathbb{E}} \def\FF{\mathbb{F}} \def\GG{\mathbb{G}} \def\HH{\mathbb{H}} \def\II{\mathbb{I}} \def\JJ{\mathbb{J}} \def\KK{\mathbb{K}} \def\LL{\mathbb{L}} \def\MM{\mathbb{M}} \def\NN{\mathbb{N}} \def\OO{\mathbb{O}} \def\PP{\mathbb{P}} \def\QQ{\mathbb{Q}} \def\RR{\mathbb{R}} \def\SS{\mathbb{S}} \def\TT{\mathbb{T}} \def\UU{\mathbb{U}} \def\VV{\mathbb{V}} \def\WW{\mathbb{W}} \def\XX{\mathbb{X}} \def\YY{\mathbb{Y}} \def\ZZ{\mathbb{Z}}  

\def\fa{\mathfrak{a}} \def\fb{\mathfrak{b}} \def\fc{\mathfrak{c}} \def\fd{\mathfrak{d}} \def\fe{\mathfrak{e}} \def\ff{\mathfrak{f}} \def\fg{\mathfrak{g}} \def\fh{\mathfrak{h}} \def\fj{\mathfrak{j}} \def\fk{\mathfrak{k}} \def\fl{\mathfrak{l}} \def\fm{\mathfrak{m}} \def\fn{\mathfrak{n}} \def\fo{\mathfrak{o}} \def\fp{\mathfrak{p}} \def\fq{\mathfrak{q}} \def\fr{\mathfrak{r}} \def\fs{\mathfrak{s}} \def\ft{\mathfrak{t}} \def\fu{\mathfrak{u}} \def\fv{\mathfrak{v}} \def\fw{\mathfrak{w}} \def\fx{\mathfrak{x}} \def\fy{\mathfrak{y}} \def\fz{\mathfrak{z}}
\def\fgl{\mathfrak{gl}}  \def\fsl{\mathfrak{sl}}  \def\fso{\mathfrak{so}}  \def\fsp{\mathfrak{sp}}  
\def\GL{\mathrm{GL}} \def\SL{\mathrm{SL}}  \def\SP{\mathrm{SL}}

\def\<{\langle} \def\>{\rangle}
\usepackage{mathabx}
\def\acts{\lefttorightarrow}
\def\ad{\mathrm{ad}} 
\def\Aut{\mathrm{Aut}}
\def\Ann{\mathrm{Ann}}
\def\dim{\mathrm{dim}} 
\def\End{\mathrm{End}} 
\def\ev{\mathrm{ev}} 
\def\Fr{\mathcal{F}\mathrm{r}}
\def\half{\hbox{$\frac12$}}
\def\Hom{\mathrm{Hom}} 
\def\id{\mathrm{id}} 
\def\sgn{\mathrm{sgn}}  
\def\supp{\mathrm{supp}}  
\def\Tor{\mathrm{Tor}}
\def\tr{\mathrm{tr}} 
\def\vep{\varepsilon}
\def\f{\varphi}


\def\Obj{\mathrm{Obj}}
\def\normeq{\unlhd}
\def\Set{{\cS\mathrm{et}}}
\def\Fin{{\cF\mathrm{inSet}}}
\def\Set{{\cS\mathrm{et}}}
\def\Grp{{\cG\mathrm{rp}}}
\def\Ab{{\cA\mathrm{b}}}
\def\Mod{{\cM\mathrm{od}}}
\def\ab{\mathrm{ab}}
\def\lcm{\mathrm{lcm}}
\def\ZZn{\ZZ/n\ZZ}


\newcommand{\ProblemID}[2]{{\def\arraystretch{1.5}
	\begin{tabular}{|lr|}\hline
	Problem: & \bf #1\\\hline
	No.\ stars:& \bf #2\\\hline\end{tabular}}}


\newcommand{\Rubric}[1]{$~$\\\vfill \hfill{\def\arraystretch{1.75}\begin{tabular} {|c|c|} \hline
#1 & Points Possible  \\ \hline \hline
complete & \hspace{3mm} 0 \hspace{3mm} 1 \hspace{3mm} 2 \hspace{3mm} 
			3 \hspace{3mm} 4 \hspace{3mm} 5 \hspace{3mm} \\ \hline
mathematically valid & \hspace{3mm} 0 \hspace{3mm} 1 \hspace{3mm} 2 \hspace{3mm} 
			3 \hspace{3mm} 4 \hspace{3mm} 5 \hspace{3mm} \\ \hline
readable/fluent & \hspace{3mm} 0 \hspace{3mm} 1 \hspace{3mm} 2 \hspace{3mm} 
			3 \hspace{3mm} 4 \hspace{3mm} 5 \hspace{3mm} \\ \hline
Total:& \qquad\qquad\qquad(out of 15)\\
\hline
\end{tabular}}
\pagebreak}




\def\NAME{Chris Hayduk}%replace "YOUR-NAME" with your full name.



\title{Final proofs portfolio}
\author{}
\usepackage{fancyhdr}
\pagestyle{fancy}
\fancyhf{}
\rhead{\NAME}
\lhead{Final proofs portfolio}
\rfoot{\thepage}


%%%%%%%%%%%%%%%%%%%%%%%%%%%%%% 
%%%%%%%%%%%%%%%%%%%%%%%%%%%%%%



\begin{document}
\begin{flushright}
\NAME
\\\smallskip
Math B4900\\
Final proofs portfolio\\
May 25, 2020
\end{flushright}

\vspace{1in}
{\def\arraystretch{1.5}
\begin{center}
\begin{tabular}{|c|c||c|c|}\hline
\textbf{Problem} & $\quad$\textbf{$\star$s}$\quad$ & \textbf{Points} & \textbf{Tot}\\\hline\hline
	% Include lines per problem that you complete (adding more if needed),
	% replacing "0X" with the problem number, and 
	% ??? with the number of stars. For example,
	% 4A 	& 2 	&&\\\hline
	1C 	& 2 	&&\\\hline
	2B 	& 2 	&&\\\hline
	3B 	& 1 	&&\\\hline
	5A 	& 2 	&&\\\hline
	6A 	& 2 	&&\\\hline
	6C 	& 2 	&&\\\hline
	8B 	& 1 	&&\\\hline
	8C 	& 2 	&&\\\hline
	9A 	& 2 	&&\\\hline
	%%%
\hline&&&$\qquad\qquad$\\\hline
\end{tabular}
\end{center}}


\vfill




\pagebreak
%%%%%%%%%%%%%%%%%%%%%%%%%%%%%%%%
%%%%%%%%% Copy and past one of these %%%%%%%%%
%%%%%%%%% for each problem you rewrite %%%%%%%%
%%%%%%%%%%%%%%%%%%%%%%%%%%%%%%%%


\hbox{\begin{minipage}{5in}
\noindent {\bf Statement:} 
Let $F$ be a field and $V$ be a vector space over $F$. Fix $\f \in \End(V)$. For $\lambda \in F$, prove that the weight space $V_\lambda$ and the generalized weight space $V^\lambda$ are both subspaces of $V$. 
\end{minipage} \hspace{.3in} {\begin{minipage}{1.1in}
\ProblemID
		{1C}%PUT PROBLEM NUMBER HERE, e.g. 4A in place of 0X.
		{2}%PUT NUMBER OF STARS HERE, e.g. 2 in place of 0.
\end{minipage}}}

\begin{proof}
Let $F$ be a field and $V$ be a vector space over $F$. Fix $\lambda \in F$. By definition, every $v \in V_{\lambda}$ is also an element of $V$. Hence, we have $V_{\lambda} \subset V$. Now observe that $\lambda 0 = 0$ for any $\lambda \in F$. Hence, $0 \in V_{\lambda}$ and thus $V_\lambda$ is non-empty. Now, by the submodule criterion, we just need to show that $x + ry \in V_{\lambda}$ for all $r \in F$ and for all $x, y \in V_{\lambda}$. Let us start by applying $\varphi$ to this element and using the properties of the linearity of $\varphi$,
\begin{align*}
\varphi(x + ry) &= \varphi(x) + r\varphi(y)\\
&= \lambda x + r \lambda y\\
&= \lambda (x + ry)
\end{align*}

Hence, $x + ry \in V_{\lambda}$ and so $V_{\lambda}$ is a subspace of $V$.\\

By definition, every $v \in V^{\lambda}$ is also an element of $V$. Hence, we have $V^{\lambda} \subset V$. Now observe that, for any $\lambda \in F$,
\begin{align*}
&\f(v) = \lambda 0 = 0v = 0\\
\iff &(\f - \lambda \cdot \id)(0) = 0
\end{align*} 

Hence, $0 \in V^{\lambda}$ and thus $V^\lambda$ is non-empty. Again, by the submodule criterion, we just need to show that $x + ry \in V^{\lambda}$ for all $r \in F$ and for all $x, y \in V^{\lambda}$. Since, $x, y \in V^{\lambda}$, we have that 
\begin{align*}
(\f - \lambda \cdot \id)^{\ell}(x) &= 0\\
(\f - \lambda \cdot \id)^{m}(y) &= 0
\end{align*}

Let $k = \max\{\ell, m\}$. Then by the fact that if $(\f - \lambda \cdot \id)^m (v) = 0$, then $(\f - \lambda \cdot \id)^n v = 0$ for all integers $n \ge m$ and by the fact that linear combinations and compositions of linear functions are linear, we have that,
\begin{align*}
(\f - \lambda \cdot \id)^k (x + ry) &= (\f - \lambda \cdot \id)^k(x) + r (\f - \lambda \cdot \id)^k(y)\\
&= 0 + r0\\
&= 0
\end{align*}

Hence, $x + ry \in V^{\lambda}$ and so $V^{\lambda}$ is also a subspace of $V$.
\end{proof}

\Rubric{}

\newpage


%%%%%%%%%%%%%%%%%%%%%%%%%%%%%%%%
%%%%%%%%% Copy and past one of these %%%%%%%%%
%%%%%%%%% for each problem you rewrite %%%%%%%%
%%%%%%%%%%%%%%%%%%%%%%%%%%%%%%%%

\hbox{\begin{minipage}{5in}
\noindent {\bf Statement:} 
Prove that determinant is invariant under change of basis. {[The details required in this proof are outlined in Homework 2; be sure to hit all the beats highlighted in that problem statement.]} 
\end{minipage} \hspace{.3in} {\begin{minipage}{1.1in}
\ProblemID
		{2B}%PUT PROBLEM NUMBER HERE, e.g. 4A in place of 0X.
		{2}%PUT NUMBER OF STARS HERE, e.g. 2 in place of 0.
\end{minipage}}}

\begin{proof}
Note that for $I_n$, we have that $\alpha_{ii} = 1$ for all $1 \leq i \leq n$ and $\alpha_{ij} = 0$ for all $i \neq j$. Hence, in the definition of the determinant, we must have that the only non-zero term in the summation is the one corresponding to the identity $\sigma = 1$. This gives us,
\begin{align*}
\det(I_n) &= \sgn(1)\alpha_{1,1}\alpha_{2,2}\cdots\alpha_{n,n}\\
&= 1 \cdot 1 \cdot 1 \cdots 1\\
&= 1
\end{align*}

Now let $A \in \GL_n(F)$. Then $A$ is invertible with inverse $A^{-1}$. But $A^{-1}$ is also invertible with inverse $A$, so $A^{-1} \in \GL_n(F)$. Hence, by fact (2) on the Lecture 4 worksheet, we have $\det(A^{-1}), \det(A)^{-1} \neq 0$. Furthermore, since $\GL_n(F) \subset M_n(F)$, we have that $A, A^{-1} \in M_n(F)$ and so we can apply fact (3) from the Lecture 4 worksheet. Thus, we have that,
\begin{align*}
\det(AA^{-1}) &= \det(I_n)\\
&= 1\\
&= \det(A)\det(A^{-1})
\end{align*}

Since $\det(A)\det(A^{-1}) = 1$, we have that $\det(A^{-1}) = \det(A)^{-1}$.\\

Now let $B \in \GL_n(F)$. Consider $\det(ABA^{-1})$ and using fact (3) along with the associativity of matrix multiplication, we get,
\begin{align*}
\det((AB)A^{-1}) &= \det(AB)\det(A^{-1})\\
&= \det(A)\det(B)\det(A^{-1})
\end{align*}

By our initial derivation, we have that $\det(A^{-1}) = \det(A)^{-1}$ and so, by the fact that $F$ is a field and hence commutative, we have,
\begin{align*}
\det((AB)A^{-1}) &= \det(A)\det(B)\det(A^{-1})\\
&= \det(A)\det(B)\det(A)^{-1}\\
&= \det(A)\det(A)^{-1}\det(B)\\
&= 1 \cdot \det(B)\\
&= \det(B)
\end{align*}

Thus, if we let $A$ be the matrix of the determinant under basis $\mathcal{A}$, and let $P$ be the change of basis matrix from $\mathcal{A}$ to some basis $\mathcal{B}$. Then $P^{-1}AP = B$, where $B$ is the matrix of the determinant under basis $\mathcal{B}$. However, from the above we get,
\begin{align*}
\det(P^{-1}AP) &= \det(A)\\
&= \det(B)
\end{align*}

Hence, the determinant is invariant under change of basis.
\end{proof}

\Rubric{}

\newpage

%%%%%%%%%%%%%%%%%%%%%%%%%%%%%%%%
%%%%%%%%% Copy and past one of these %%%%%%%%%
%%%%%%%%% for each problem you rewrite %%%%%%%%
%%%%%%%%%%%%%%%%%%%%%%%%%%%%%%%%

\hbox{\begin{minipage}{5in}
\noindent {\bf Statement:} 
Let $X \in M_n(\CC)$, let $\Lambda$ be the set of eigenvalues for $X$, and let $m_\lambda$ be the multiplicity of $\lambda \in \Lambda$. Show 
$$\tr(X) = \sum_{\lambda \in \Lambda} \lambda m_\lambda
	\quad \text{ and } \quad 
	\det(X) = \prod _{\lambda \in \Lambda} \lambda^{m_\lambda}.$$
\end{minipage} \hspace{.3in} {\begin{minipage}{1.1in}
\ProblemID
		{3B}%PUT PROBLEM NUMBER HERE, e.g. 4A in place of 0X.
		{1}%PUT NUMBER OF STARS HERE, e.g. 2 in place of 0.
\end{minipage}}}

\begin{proof}
Since $\CC$ is an algebraically closed field and $X \in M_n(\CC)$, we have that there is some $J$ in Jordan canonical form such that $J \sim X$. That is, there is some choice of basis under which $X$ can be written in Jordan canonical form. Trace is invariant under choice of basis, so we have that $\tr(X) = \tr(J)$. Note that in Jordan canonical form, the eigenvalues of $X$ are placed along the diagonal. Thus, the diagonal of $J$ contains all of the eigenvalues of $X$. Moreover, the multiplicity of an eigenvalue is given by the number of rows in which it appears in the matrix $J$. Hence, we must have that,
\begin{align*}
\tr(X) &= \tr(J)\\
&= \sum_{\lambda \in \Lambda} \lambda m_{\lambda}
\end{align*}

as required.


Similarly to the above, we have $\det(X) = \det(J)$. Now consider the characteristic polynomial of $J$. This is given by,

\begin{align*}
c_J(x) &= \det(J - x \cdot \id)\\
&= \prod_{\lambda \in \Lambda} (\lambda - x)^{m_{\lambda}}
\end{align*}

If we plug in $0$ for $x$, we get,
\begin{align*}
c_J(0) &= \det(J - 0 \cdot \id)\\
&= \det(J)\\
&= \prod_{\lambda \in \Lambda} (\lambda)^{m_{\lambda}}
\end{align*}

Hence, we have that,
\begin{align*}
\det(X) &= \det(J)\\
&= \prod_{\lambda \in \Lambda} (\lambda)^{m_{\lambda}}
\end{align*}

as required.
\end{proof}

\Rubric{}

\newpage

%%%%%%%%%%%%%%%%%%%%%%%%%%%%%%%%
%%%%%%%%% Copy and past one of these %%%%%%%%%
%%%%%%%%% for each problem you rewrite %%%%%%%%
%%%%%%%%%%%%%%%%%%%%%%%%%%%%%%%%

\hbox{\begin{minipage}{5in}
\noindent {\bf Statement:} 
Prove that $\Hom_A(*,M)$ is an exact functor. Namely, show that if $0 \hookrightarrow X \xrightarrow{f} Y \xrightarrow{g} Z \to 0$ is a split exact sequence of $A$-modules, then so is 
$$0 \hookrightarrow \Hom_A(M,X) \xrightarrow{F} \Hom_A(M, Y) \xrightarrow{G} \Hom_A(M,Z) \to 0,$$
where $F(\f) = f \circ \f$ and $G(\f) = g \circ \f$. {[You may use any other propositions or theorems from Lecture 10 or before.]}
\end{minipage} \hspace{.3in} {\begin{minipage}{1.1in}
\ProblemID
		{5A}%PUT PROBLEM NUMBER HERE, e.g. 4A in place of 0X.
		{2}%PUT NUMBER OF STARS HERE, e.g. 2 in place of 0.
\end{minipage}}}

\begin{proof}
Since $0 \hookrightarrow X \xrightarrow{f} Y \xrightarrow{g} Z$ is exact, by Proposition 2.2 from $\S$ III.2 in Lang, we have that $$0 \hookrightarrow \Hom_A(M,X) \xrightarrow{F} \Hom_A(M, Y) \xrightarrow{G} \Hom_A(M,Z)$$ is exact for all $A$-modules $M$. In order to show that $$0 \hookrightarrow \Hom_A(M,X) \xrightarrow{F} \Hom_A(M, Y) \xrightarrow{G} \Hom_A(M,Z) \to 0$$ is a short exact sequence, we thus need to show that $G$ is surjective. Fix $\varphi \in \Hom_A(M, Z)$. We want to show that there is a $\varphi' \in \Hom_A(M, Y)$ such that $g \circ \varphi' = \varphi$. Observe that since $Y \xrightarrow{g} Z$, we have that $g$ is surjective. Hence, $g$ maps onto every element of $Z$. Since the domain of $G$ is the set of all homomorphisms from $M \to Y$, we can take any valid mapping of elements in $M$ to elements in $Y$. In particular, since $g$ is surjective, we can select a mapping from $\varphi'(m) \mapsto y$ such that $g(\varphi(m)) = g(y) = z$ for any $z \in Z$. Hence, for a fixed $\varphi: M \to Z$, let us select $\varphi'$ such that $\varphi'(m) \mapsto y$ such that $g(\varphi'(m)) = \varphi(m)$ for all $m \in M$. Hence, we have that $G(\varphi')= g \circ \varphi' = \varphi$. Since $\varphi$ was arbitrary, we thus have that $G$ is surjective. Thus, $$0 \hookrightarrow \Hom_A(M,X) \xrightarrow{F} \Hom_A(M, Y) \xrightarrow{G} \Hom_A(M,Z) \to 0$$ is an exact sequence. Now we must show that the sequence above is split. Note that, by $\S 10.5$, Proposition 29 in Dummit and Foote, we have that since $0 \hookrightarrow X \xrightarrow{f} Y \xrightarrow{g} Z \to 0$ is a split sequence, then so is $0 \hookrightarrow \Hom_A(M,X) \xrightarrow{F} \Hom_A(M, Y) \xrightarrow{G} \Hom_A(M,Z) \to 0$. Hence, we have shown that this sequence is both exact and split, and so it is a split exact sequence. Since $0 \hookrightarrow X \xrightarrow{f} Y \xrightarrow{g} Z \to 0$ is a split exact sequence, there exists a function $\lambda: Y \to X$ such that $\lambda f = \id$. Let us define $\Lambda: \Hom_A(M, Y) \to \Hom_A(M, X)$ by $\varphi \mapsto \lambda \circ \varphi$. Thus, for $\varphi \in \Hom_A(M, Y)$, we have,
\begin{align*}
(F \circ \Lambda)(\varphi) &= F(\Lambda(\varphi))\\
&= F(\lambda \circ \varphi)\\
&= f \circ (\lambda \circ \varphi)\\
&= (f \circ \lambda) \circ \varphi\\
&= \id \circ \varphi\\
&= \varphi
\end{align*}

Thus, we have that $F \circ \Lambda = \id$ and so $\Lambda$ is a splitting homomorphism and hence, $$0 \hookrightarrow \Hom_A(M,X) \xrightarrow{F} \Hom_A(M, Y) \xrightarrow{G} \Hom_A(M,Z) \to 0,$$

is a split exact sequence.
\end{proof}

\Rubric{}

\newpage

%%%%%%%%%%%%%%%%%%%%%%%%%%%%%%%%
%%%%%%%%% Copy and past one of these %%%%%%%%%
%%%%%%%%% for each problem you rewrite %%%%%%%%
%%%%%%%%%%%%%%%%%%%%%%%%%%%%%%%%

\hbox{\begin{minipage}{5in}
\noindent {\bf Statement:} 
Prove that $M$ is simple if and only if $Am = M$ for any non-zero $m \in M$. 
\end{minipage} \hspace{.3in} {\begin{minipage}{1.1in}
\ProblemID
		{6A}%PUT PROBLEM NUMBER HERE, e.g. 4A in place of 0X.
		{2}%PUT NUMBER OF STARS HERE, e.g. 2 in place of 0.
\end{minipage}}}

\begin{proof}
Suppose $M$ is simple. Then the only submodules of $M$ are $0$ and itself. We have that $Am$ is a submodule of $M$ for any $m \neq 0 \in M$. Since $M$ is simple, we have that either $Am = 0$ or $Am = M$. However, we know that $m \neq 0$. Since $A$ is a ring with $1$, we have that $1m = m \in Am$ and so $Am \neq 0$. Thus, we must have that $Am = M$ for any non-zero $m \in M$.\\

Now suppose that $Am = M$ for any non-zero $m \in M$. Let $N \subset M$ be a submodule and suppose $N \neq 0$. Thus there is some $m \in N \setminus \{0\} \subset M \setminus \{0\}$ . Now since $Am = M$ for every non-zero $m \in M$, we must have that $Am = M$ for this particular choice of $m$. Since $N$ is a submodule and closed under the action of $A$ on $N$, we must have that $N = M$. Thus, $M$ is simple.
\end{proof}

\Rubric{}

\newpage

%%%%%%%%%%%%%%%%%%%%%%%%%%%%%%%%%%%%%%%%
%%%%%%%%%%%%%%%%%%%%%%%%%%%%%%%%%%%%%%%%
%%%%%%%%%%%%%%%%%%%%%%%%%%%%%%%%%%%%%%%%
%%%%%%%%%%%%%%%%%%%%%%%%%%%%%%%%%%%%%%%%

\hbox{\begin{minipage}{5in}
\noindent {\bf Statement:} 
Show that if $A$ is a commutative ring with 1, that $A^m \cong A^n$ if and only if $n=m$.
\end{minipage} \hspace{.3in} {\begin{minipage}{1.1in}
\ProblemID
		{6C}%PUT PROBLEM NUMBER HERE, e.g. 4A in place of 0X.
		{2}%PUT NUMBER OF STARS HERE, e.g. 2 in place of 0.
\end{minipage}}}

\begin{proof}
Suppose $A$ is is a commutative ring with $1$ and suppose that $A^m \cong A^n$. Let $I$ be a maximal ideal of $A$. Since $A^m \cong A^n$, we have that $IA^m \cong IA^n$, and so $A^m/IA^m \cong A^n/IA^n$. Moreover, on Q3 of Homework 6, we proved that for a free module $M$ over a ring $A$ with basis $\mathcal{B}$ and ideal $I$, we have that,
\begin{align*}
M/IM \cong \bigoplus_{b \in \mathcal{B}} Ab/Ib
\end{align*}

If we fix $A$ as our ring with $A^m, A^n$ as the free modules over this ring, we can apply this result as follows,
\begin{align*}
A^m/IA^m &\cong \oplus_{b \in \mathcal{B}} Ab/Ib\\
A^n/IA^n &\cong \oplus_{c \in \mathcal{C}} Ac/Ic
\end{align*}

Thus, we have that $\oplus_{b \in \mathcal{B}} Ab/Ib \cong \oplus_{c \in \mathcal{C}} Ac/Ic$. Since $\oplus_{b \in \mathcal{B}} Ab/Ib$ is $|\mathcal{B}|$-dimensional and $\oplus_{c \in \mathcal{C}} Ac/Ic$ is $|\mathcal{C}|$-dimensional, we must have that $|\mathcal{B}| = |\mathcal{C}|$. But note that $\mathcal{B}$ is a basis for $A^m$ and $\mathcal{C}$ is a basis for $A^n$. We know that $A^m$ must have a basis of size $m$ and $A^n$ must have a basis of size $n$, so let us take $\mathcal{B}$ and $\mathcal{C}$ to be of these sizes, respectively. Thus, we have that $|\mathcal{B}| = |\mathcal{C}|$ implies that $n = m$, as required.\\

Now suppose $n = m$. Then we must have $A^m = A^n$ and so $A^m \cong A^n$ by the identity map.
\end{proof}

\Rubric{}

\newpage

%%%%%%%%%%%%%%%%%%%%%%%%%%%%%%%%%%%%%%%%
%%%%%%%%%%%%%%%%%%%%%%%%%%%%%%%%%%%%%%%%
%%%%%%%%%%%%%%%%%%%%%%%%%%%%%%%%%%%%%%%%
%%%%%%%%%%%%%%%%%%%%%%%%%%%%%%%%%%%%%%%%

\hbox{\begin{minipage}{5in}
\noindent {\bf Statement:} 
Classify the semisimple $\ZZ$-modules.
\end{minipage} \hspace{.3in} {\begin{minipage}{1.1in}
\ProblemID
		{8B}%PUT PROBLEM NUMBER HERE, e.g. 4A in place of 0X.
		{1}%PUT NUMBER OF STARS HERE, e.g. 2 in place of 0.
\end{minipage}}}

\begin{proof}
Note that a $\ZZ$ module is simple if it is of the form $\ZZ/I$ where $I$ is a maximal ideal. The ideals of $\ZZ$ are precisely the sets of all integers divisible by a fixed integer $n$. That is, $n\ZZ$ is an ideal for all $n \in \ZZ$. Recall that an ideal $n\ZZ$ of $\ZZ$ is maximal if there are no other ideals of the form $k\ZZ$ such that $n\ZZ \subset k\ZZ \subset \ZZ$. Observe that if $n$ is a composite integer, then we can write $n = p_1p_2 \cdots p_{\ell}$ for primes in $\ZZ$. That is, for any $p_j$ in that expansion, we have that $p_j$ divides $n$ and thus all multiples of $n$. Hence, $n\ZZ \subset p_j \ZZ$ for any prime $p_j$ in that expansion. Moreover, for every prime we must have that there is no integer $m$ such that $p_j \ZZ \subset m \ZZ$, otherwise $m$ would divide $p_j$ and hence $p_j$ would not be prime. Thus, the maximal ideals of $\ZZ$ are precisely of the form $p\ZZ$ where $p$ is a prime.\\

Now we have that the simple modules of $\ZZ$ are of the form $\ZZ/p\ZZ$ for all primes $p \in \ZZ$. Since semisimple modules are direct sums of simple modules, we have that any semisimple module of $\ZZ$ is of the form:
\begin{align*}
p_1 \ZZ \oplus p_2 \ZZ \oplus \cdots \oplus p_{\ell} \ZZ
\end{align*}

for some primes $p_1, \ldots, p_{\ell}$ (not necessarily distinct).
\end{proof}

\Rubric{}

\newpage

%%%%%%%%%%%%%%%%%%%%%%%%%%%%%%%%%%%%%%%%
%%%%%%%%%%%%%%%%%%%%%%%%%%%%%%%%%%%%%%%%
%%%%%%%%%%%%%%%%%%%%%%%%%%%%%%%%%%%%%%%%
%%%%%%%%%%%%%%%%%%%%%%%%%%%%%%%%%%%%%%%%

\hbox{\begin{minipage}{5in}
\noindent {\bf Statement:} 
Let $M$ be a semisimple $A$-module. Prove that the following are equivalent: 
\begin{enumerate}[(i)]
\item $M$ is finitely-generated;
\item $M$ is Noetherian;
\item $M$ is Artinian;
\item $M$ is a finite direct sum of simple modules. 
\end{enumerate}
\end{minipage} \hspace{.3in} {\begin{minipage}{1.1in}
\ProblemID
		{8C}%PUT PROBLEM NUMBER HERE, e.g. 4A in place of 0X.
		{2}%PUT NUMBER OF STARS HERE, e.g. 2 in place of 0.
\end{minipage}}}

\begin{proof}
First we will show that (i) is equivalent to (ii). Suppose $M$ is finitely generated. Then there exist $m_1, m_2, \ldots, m_n \in M$ such that for any $x \in M$, there exist $a_1, a_2, \ldots, a_n \in A$ with $x = a_1m_1 + a_2m_2 + \cdots + a_nm_n$. Since every element of a submodule of $M$ is also an element of $M$, then it must be true that every element of a submodule $N$ of $M$ is finitely generated as well. Hence, $M$ is Noetherian. Now suppose $M$ is Noetherian. Then every submodule of $M$ is finitely generated. In particular, since $M$ is a submodule of itself, it must be finitely generated. Thus, (i) and (ii) are equivalent.\\

Now we will show the equivalence of (i) and (iv). Suppose $M$ is semisimple and finitely generated. Then $M$ is the direct sum of simple modules and, since (i) is equivalent to (ii), each of those submodules is finitely generated. Since the generators of $M$ are finite, they can old by combined in a finite number of ways. Hence, there must be finitely many submodules which are finitely generated. Hence, $M$ is a finite direct sum of simple modules. Now let us assume that $M$ is a finite direct sum of simple modules and work towards the other directions. Every simple module is cyclic and hence generated by one element. The union of these generators forms a basis for $M$ since $M$ is a direct sum of these simple modules. Since there are a finite number of these simple modules, then $M$ is finitely-generated by this union as required. Hence, by this and our previous work, (i), (ii), and (iv) are equivalent.\\

Now we will show the equivalence of (iii) and (iv). We will start by showing the contrapositive of (iii) implies (iv) [not (iv) implies not (iii)]. Suppose $M$ is not a finite direct sum of simple modules. Since $M$ is semisimple, we can write it as $M \cong \oplus_{\lambda} M_{\lambda}$ where $M_{\lambda}$ is simple. Equivalently, we have $M = \sum_{\lambda \in \Lambda} M_{\lambda}$. Since we have assumed that $M$ is not a finite direct sum of simple modules, we must have that $\Lambda$ is infinite. Let $\tilde{\Lambda} = \{\lambda_1, \lambda_2, \ldots\}$ be a countable subset of $\Lambda$. Let us now define, $$N_0 = \sum_{\lambda_i = \tilde{\Lambda}} M_{\lambda_i}$$ and $$N_i = \sum_{\lambda \in \tilde{\Lambda} \setminus \{\lambda_1, \ldots, \lambda_i\}} M_{\lambda}$$ We have thus constructed an infinite descending chain of subsets of $M$ such that $$M \supset N_0 \supset N_1 \supset \cdots$$ since $N_{i+1}$ is a proper subset of $N_i$ for all $i$. Hence, we have that $M$ is not Artinian, as required.\\

Now we will show that (iii) implies (iv). Again consider that, since $M$ is semisimple, we have $M \cong \oplus_{\lambda} M_{\lambda}$. Let $\tilde{\Lambda} = \{\lambda_1, \lambda_2, \ldots\} \subset \Lambda$ be a well-ordered subset, Again, we will define $N_0$ and $N_i$ exactly as above. Since we have assumed that $M$ is Artinian, we must have that the following sequence of subsets of $M$ terminates, $$N_0 \supset N_1 \supset N_2 \supset \cdots$$ Since this sequence terminates, and we have that $N_{i+1}$ is a proper subset of $N_i$ for all $i$, we must have that the above sequence is finite. Thus, $\tilde{\Lambda}$ is finite and, as a result, we have that $\Lambda$ is finite, as required. Hence, we have shown that all four properties are equivalent.
\end{proof}

\Rubric{}

\newpage

%%%%%%%%%%%%%%%%%%%%%%%%%%%%%%%%%%%%%%%%
%%%%%%%%%%%%%%%%%%%%%%%%%%%%%%%%%%%%%%%%
%%%%%%%%%%%%%%%%%%%%%%%%%%%%%%%%%%%%%%%%
%%%%%%%%%%%%%%%%%%%%%%%%%%%%%%%%%%%%%%%%

\hbox{\begin{minipage}{5in}
\noindent {\bf Statement:} 
Let $M$ be a completely reducible $A$-module. Show that for any submodule $N \subseteq M$, we have $M/N$ is completely reducible as well. Moreover, if 
$$M \cong \bigoplus_{\lambda \in \Lambda}M_\lambda, \quad \text{then} \quad 
	M/N \cong \bigoplus_{\lambda \in \Gamma}M_\lambda,$$
	for some $\Gamma \subseteq \Lambda$.
\end{minipage} \hspace{.3in} {\begin{minipage}{1.1in}
\ProblemID
		{9A}%PUT PROBLEM NUMBER HERE, e.g. 4A in place of 0X.
		{2}%PUT NUMBER OF STARS HERE, e.g. 2 in place of 0.
\end{minipage}}}

\begin{proof}
Since $M$ is completely reducible, we have that $M \cong \bigoplus_{\lambda \in \Lambda}M_\lambda$ where $M_{\lambda}$ is simple. This is equivalent to $M = \sum_{\lambda \in \Lambda} M_{\lambda}$ with $\left(\sum_{\lambda \in \Lambda - \mu} M_{\lambda} \right) \cap M_{\mu} = 0$. Hence, for any $m \in M$, we have that $m = \sum_{\lambda \in \Lambda} m_{\lambda}$ uniquely.\\

Let $N \subset M$. Then,
\begin{align*}
M/N &= \left(\sum_{\lambda \in \Lambda} M_{\lambda} \right)/N\\
&= \{m + N \ | \ m \in \sum_{\lambda \in \Lambda \text{; finite}} M_{\lambda}\}
\end{align*}

Thus, for $m + N \in M/N$, we have,
\begin{align*}
m + N &= \sum_{\lambda \text{; finite}} m_{\lambda} + N\\
&= \sum_{\lambda \text{; finite}} m_{\lambda} + \sum_{\lambda; m_{\lambda} \neq 0} N\\
&= \{\sum_{\lambda \text{; } m_{\lambda} \neq 0} n_{\lambda} \ | \ n_{\lambda} \in N\}\\
&= N\\
&= \sum_{\lambda \text{; finite}} (m_{\lambda} + N) \in (M_{\lambda} + N)/N
\end{align*}

The above derivation thus gives us that $M/N = \sum_{\lambda} (M_{\lambda} + N)/N$. Now, applying the second isomorphism theorem for modules, we have that,
\begin{align*}
M/N &= \sum_{\lambda} (M_{\lambda} + N)/N\\
&= \sum_{\lambda} M_{\lambda}/(M_{\lambda} \cap N)
\end{align*}

Observe that, since each $M_{\lambda}$ is simple, we have that $M_{\lambda} \cap N = 0$ if $M_{\lambda} \not\subset N$ and $M_{\lambda} \cap N = M_{\lambda}$ if $M_{\lambda} \subset N$. These are the only two possible values for $M_{\lambda} \cap N$. Thus, for a fixed $M_{\lambda}$, we have either that,
\begin{align*}
M_{\lambda}/(M_{\lambda} \cap N) &= M_{\lambda}/0\\
&= \{m_{\lambda} + 0 \ | \ m_{\lambda} \in M_{\lambda}\}\\
&= M_{\lambda}
\end{align*}

or,
\begin{align*}
M_{\lambda}/(M_{\lambda} \cap N) &= M_{\lambda}/M_{\lambda}\\
&= 0
\end{align*}

Thus, we have that $\sum_{\lambda} M_{\lambda}/(M_{\lambda} \cap N)$ corresponds to some subset $\Gamma \subset \Lambda$, since the terms are either $M_{\lambda}$ for some $\lambda$ or $0$. This gives us that,
\begin{align*}
M/N &= \sum_{\lambda} (M_{\lambda} + N)/N\\
&= \sum_{\lambda} M_{\lambda}/(M_{\lambda} \cap N)\\
&= \sum_{\lambda \in \Gamma} M_{\lambda}\\
&\cong \bigoplus_{\lambda \in \Gamma}M_\lambda
\end{align*}

as required.
\end{proof}

\Rubric{}

\end{document}