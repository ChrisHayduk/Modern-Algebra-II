\documentclass[11pt, reqno]{amsart}
\usepackage[margin=1in]{geometry}    
\geometry{letterpaper}       
%\geometry{landscape}                % Activate for for rotated page geometry
\usepackage[parfill]{parskip}    % Activate to begin paragraphs with an empty line rather than an indent
\usepackage{amsfonts, amscd, amssymb, amsthm, amsmath}
\usepackage{pdfsync}  %leaves makers for tex searching
\usepackage{enumerate}
\usepackage[pdftex,bookmarks]{hyperref}



%%% Theorems %%%--------------------------------------------------------- 
\theoremstyle{plain}
	\newtheorem{thm}{Theorem}[section]
	\newtheorem{lemma}[thm]{Lemma}
	\newtheorem{prop}[thm]{Proposition}
	\newtheorem{cor}[thm]{Corollary}
\theoremstyle{definition}
	\newtheorem*{defn}{Definition}
	\newtheorem{remark}[thm]{Remark}
\theoremstyle{example}
	\newtheorem*{example}{Example}


%%% Environments %%%--------------------------------------------------------- 
\newenvironment{ans}{\medskip \paragraph*{\emph{Answer}.}}{\hfill \break  $~\!\!$ \dotfill \medskip }
\newenvironment{sketch}{\medskip \paragraph*{\emph{Proof sketch}.}}{ \medskip }
\newenvironment{summary}{\medskip \paragraph*{\emph{Summary}.}}{  \hfill \break  \rule{1.5cm}{0.4pt} \medskip }
\newcommand\Ans[1]{\hfill \emph{Answer:} {#1}}


%%% Pictures %%%--------------------------------------------------------- 
%%% If you need to draw pictures, tikzpicture is one good option. Here are some basic things I always use:
\usepackage{tikz}
\usetikzlibrary{arrows}
\usetikzlibrary{shapes}
\tikzstyle{V}=[draw, fill =black, circle, inner sep=0pt, minimum size=2pt]
\tikzstyle{bV}=[draw, fill =black, circle, inner sep=0pt, minimum size=4pt]
\newcommand\TikZ[1]{\begin{matrix}\begin{tikzpicture}#1\end{tikzpicture}\end{matrix}}





%%% Color  %%%---------------------------------------------------------
\usepackage{color}
\newcommand{\NOTE}[1]{{\color{blue}#1}}
\newcommand{\blue}[1]{{\color{blue}#1}}
\newcommand{\red}[1]{{\color{red}#1}}
\newcommand{\MOVED}[1]{{\color{gray}#1}}
\definecolor{dred}{rgb}{.8,0,.1}
\definecolor{dgreen}{rgb}{0,.6,.1}
\definecolor{purple}{rgb}{.6,0,.8}
\definecolor{dorange}{rgb}{.8,.25,0}
\definecolor{dyellow}{rgb}{.95,.85,0}
\definecolor{alert}{rgb}{.8,.25,0}

\newcommand{\gV}[1]{\TikZ{\node[thick, dgreen, rounded corners, draw] at (0,0){#1};}}
\newcommand{\pV}[1]{\TikZ{\node[thick, purple, rounded corners, draw] at (0,0){#1};}}


%%% Alphabets %%%---------------------------------------------------------
%%% Some shortcuts for my commonly used special alphabets and characters.
\def\cA{\mathcal{A}}\def\cB{\mathcal{B}}\def\cC{\mathcal{C}}\def\cD{\mathcal{D}}\def\cE{\mathcal{E}}\def\cF{\mathcal{F}}\def\cG{\mathcal{G}}\def\cH{\mathcal{H}}\def\cI{\mathcal{I}}\def\cJ{\mathcal{J}}\def\cK{\mathcal{K}}\def\cL{\mathcal{L}}\def\cM{\mathcal{M}}\def\cN{\mathcal{N}}\def\cO{\mathcal{O}}\def\cP{\mathcal{P}}\def\cQ{\mathcal{Q}}\def\cR{\mathcal{R}}\def\cS{\mathcal{S}}\def\cT{\mathcal{T}}\def\cU{\mathcal{U}}\def\cV{\mathcal{V}}\def\cW{\mathcal{W}}\def\cX{\mathcal{X}}\def\cY{\mathcal{Y}}\def\cZ{\mathcal{Z}}

\def\AA{\mathbb{A}} \def\BB{\mathbb{B}} \def\CC{\mathbb{C}} \def\DD{\mathbb{D}} \def\EE{\mathbb{E}} \def\FF{\mathbb{F}} \def\GG{\mathbb{G}} \def\HH{\mathbb{H}} \def\II{\mathbb{I}} \def\JJ{\mathbb{J}} \def\KK{\mathbb{K}} \def\LL{\mathbb{L}} \def\MM{\mathbb{M}} \def\NN{\mathbb{N}} \def\OO{\mathbb{O}} \def\PP{\mathbb{P}} \def\QQ{\mathbb{Q}} \def\RR{\mathbb{R}} \def\SS{\mathbb{S}} \def\TT{\mathbb{T}} \def\UU{\mathbb{U}} \def\VV{\mathbb{V}} \def\WW{\mathbb{W}} \def\XX{\mathbb{X}} \def\YY{\mathbb{Y}} \def\ZZ{\mathbb{Z}}  

\def\fa{\mathfrak{a}} \def\fb{\mathfrak{b}} \def\fc{\mathfrak{c}} \def\fd{\mathfrak{d}} \def\fe{\mathfrak{e}} \def\ff{\mathfrak{f}} \def\fg{\mathfrak{g}} \def\fh{\mathfrak{h}} \def\fj{\mathfrak{j}} \def\fk{\mathfrak{k}} \def\fl{\mathfrak{l}} \def\fm{\mathfrak{m}} \def\fn{\mathfrak{n}} \def\fo{\mathfrak{o}} \def\fp{\mathfrak{p}} \def\fq{\mathfrak{q}} \def\fr{\mathfrak{r}} \def\fs{\mathfrak{s}} \def\ft{\mathfrak{t}} \def\fu{\mathfrak{u}} \def\fv{\mathfrak{v}} \def\fw{\mathfrak{w}} \def\fx{\mathfrak{x}} \def\fy{\mathfrak{y}} \def\fz{\mathfrak{z}}
\def\fgl{\mathfrak{gl}}  \def\fsl{\mathfrak{sl}}  \def\fso{\mathfrak{so}}  \def\fsp{\mathfrak{sp}}  
\def\GL{\mathrm{GL}} \def\SL{\mathrm{SL}}  \def\SP{\mathrm{SL}}

\def\<{\langle} \def\>{\rangle}
\def\({\<\!\<}\def\){\>\!\>}
\def\ad{\mathrm{ad}} 
\def\Aut{\mathrm{Aut}}
\def\dim{\mathrm{dim}} 
\def\End{\mathrm{End}} 
\def\ev{\mathrm{ev}} 
\def\half{\hbox{$\frac12$}}
\def\img{\mathrm{img}}
\def\Ind{\mathrm{Ind}}
\def\Hom{\mathrm{Hom}} 
\def\Fn{\mathrm{Fn}} 
\def\Fr{\mathcal{F}\mathrm{r}}
\def\hgt{\mathrm{ht}} 
\def\id{\mathrm{id}} 
\def\qtr{\mathrm{qtr}} 
\def\sgn{\mathrm{sgn}}
\def\supp{\mathrm{supp}}
\def\tr{\mathrm{tr}} 
\def\Tor{\mathrm{Tor}} 
\def\vep{\varepsilon}
\def\f{\varphi}



\def\Obj{\mathrm{Obj}}
\def\normeq{\unlhd}
\def\Set{{\cS\mathrm{et}}}
\def\Fin{{\cF\mathrm{inSet}}}
\def\Set{{\cS\mathrm{et}}}
\def\Grp{{\cG\mathrm{rp}}}
\def\Ab{{\cA\mathrm{b}}}
\def\Mod{{\cM\mathrm{od}}}
\def\ab{\mathrm{ab}}

% Arrows:
\newcommand\xdhrightarrow[2][]{%
  \mathrel{\ooalign{$\xrightarrow[#1\mkern4mu]{#2\mkern4mu}$\cr%
  \hidewidth$\rightarrow\mkern4mu$}}
}
%\newcommand\dhrightarrow{%
%  \mathrel{\ooalign{$\rightarrow$\cr%
%  $\mkern3.5mu\rightarrow$}}
%}
\def\dhrightarrow{\twoheadrightarrow}
\def\dhleftarrow{\twoheadleftarrow}


%%%%%%%%%%%%%%%%%%%%%%%%%%%%%% 
%%%%%%%%%%%%%%%%%%%%%%%%%%%%%%

\def\HW{9}
\def\DUE{4/30/2021}

\title[Homework \HW]{Homework \HW \\
Math B4900\\
\small Due: \DUE}
\author{}
%\date{}                                           % Activate to display a given date or no date

\begin{document}
%\maketitle %%% COMMENT THIS OUT and UNCOMMENT the following to give yourself a good assignment header:
\begin{flushright}
Chris Hayduk\\
Math B4900\\
Homework \HW\\
\DUE
\end{flushright}


Let $A$ be a ring with 1.
\begin{enumerate}[1.]
\item Let $M$ be a completely reducible $A$-module. Show that for any submodule $N \subseteq M$, we have $M/N$ is completely reducible as well. Moreover, if 
$$M \cong \bigoplus_{\lambda \in \Lambda}M_\lambda, \quad \text{then} \quad 
	M/N \cong \bigoplus_{\lambda \in \Gamma}M_\lambda,$$
	for some $\Gamma \subseteq \Lambda$.

{[\emph{Hint:} if $M \cong \bigoplus_{\lambda \in \Lambda}M_\lambda$ (with $M_\lambda$ simple), then, more simply, $M = \sum_{\lambda \in \Lambda} M_\lambda$ (identifying $M_\lambda$ with $\hat{M}_\lambda$). Show that $M/N = \sum_\lambda (M_\lambda + N)/N$ (write out the cosets!), and then use the second isomorphism theorem on each piece. Finally, check that, for all $\mu \in \Lambda$, we have 
$$(M_\mu +N)/N \cap \sum_{\lambda \neq \mu} (M_\lambda + N)/N = 0.]$$}

\begin{proof}
Since $M$ is completely reducible, we have that $M \cong \bigoplus_{\lambda \in \Lambda}M_\lambda$ where $M_{\lambda}$ is simple. This is equivalent to $M = \sum_{\lambda \in \Lambda} M_{\lambda}$ with $\left(\sum_{\lambda \in \Lambda - \mu} M_{\lambda} \right) \cap M_{\mu} = 0$. Hence, for any $m \in M$, we have that $m = \sum_{\lambda \in \Lambda} m_{\lambda}$ uniquely.\\

Let $N \subset M$. Then,
\begin{align*}
M/N &= \left(\sum_{\lambda \in \Lambda} M_{\lambda} \right)/N\\
&= \{m + N \ | \ m \in \sum_{\lambda \in \Lambda \text{; finite}} M_{\lambda}\}
\end{align*}

Thus, for $m + N \in M/N$, we have,
\begin{align*}
m + N &= \sum_{\lambda \text{; finite}} m_{\lambda} + N\\
&= \sum_{\lambda \text{; finite}} m_{\lambda} + \sum_{\lambda; m_{\lambda} \neq 0} N\\
&= \{\sum_{\lambda \text{; } m_{\lambda} \neq 0} n_{\lambda} \ | \ n_{\lambda} \in N\}\\
&= N\\
&= \sum_{\lambda \text{; finite}} (m_{\lambda} + N) \in (M_{\lambda} + N)/N
\end{align*}

The above derivation thus gives us that $M/N = \sum_{\lambda} (M_{\lambda} + N)/N$. Now, applying the second isomorphism theorem for modules, we have that,
\begin{align*}
M/N &= \sum_{\lambda} (M_{\lambda} + N)/N\\
&= \sum_{\lambda} M_{\lambda}/(M_{\lambda} \cap N)
\end{align*}

Observe that, since each $M_{\lambda}$ is simple, we have that $M_{\lambda} \cap N = 0$ if $M_{\lambda} \not\subset N$ and $M_{\lambda} \cap N = M_{\lambda}$ if $M_{\lambda} \subset N$. These are the only two possible values for $M_{\lambda} \cap N$. Thus, for a fixed $M_{\lambda}$, we have either that,
\begin{align*}
M_{\lambda}/(M_{\lambda} \cap N) &= M_{\lambda}/0\\
&= \{m_{\lambda} + 0 \ | \ m_{\lambda} \in M_{\lambda}\}\\
&= M_{\lambda}
\end{align*}

or,
\begin{align*}
M_{\lambda}/(M_{\lambda} \cap N) &= M_{\lambda}/M_{\lambda}\\
&= 0
\end{align*}

Thus, we have that $\sum_{\lambda} M_{\lambda}/(M_{\lambda} \cap N)$ corresponds to some subset $\Gamma \subset \Lambda$, since the terms are either $M_{\lambda}$ for some $\lambda$ or $0$. This gives us that,
\begin{align*}
M/N &= \sum_{\lambda} (M_{\lambda} + N)/N\\
&= \sum_{\lambda} M_{\lambda}/(M_{\lambda} \cap N)\\
&= \sum_{\lambda \in \Gamma} M_{\lambda}\\
&\cong \bigoplus_{\lambda \in \Gamma}M_\lambda
\end{align*}

as required.
\end{proof}

\item Let $\{z_\lambda ~|~ \lambda \in \Lambda\}$ be the centrally primitive idempotents in a semisimple ring $A$, and let $U_\lambda$ be the simple $A$-module corresponding to $\lambda \in \Lambda$. Let $M$ be an $A$-module (not necessarily the left-regular module. Use Artin-Wedderburn to show that $z_\lambda M \cong \bigoplus_{i \in \cI} U_\lambda$ (i.e.\ $z_\lambda$ projects onto a (not necessarily finite) direct sum of a bunch of copies of $U_\lambda$---called the \emph{$\lambda$-isotypic component} of $M$). 

\item Let $V = \CC^2 = \CC\{v_1, v_2\}$. Let  $\CC D_8$ act on $V^{\otimes 4} = V \otimes V \otimes V\otimes V$ by identifying the copies of $V$ with the vertices of the square, and applying the corresponding factor permutation:
$${\def\arraystretch{1.3}
\begin{array}{l}
r \cdot (v_{i_1} \otimes v_{i_2} \otimes v_{i_3} \otimes v_{i_4}) = v_{i_2} \otimes v_{i_3} \otimes v_{i_4} \otimes v_{i_1}\\
\text{and} \\
s \cdot (v_{i_1} \otimes v_{i_2} \otimes v_{i_3} \otimes v_{i_4}) = v_{i_2} \otimes v_{i_1} \otimes v_{i_4} \otimes v_{i_3}\end{array}}
\qquad\qquad 
\TikZ{
\draw (0,0) node[V, label={above left, inner sep=1pt}:{\tiny$i_1$}]{} to 
	(1,0) node[V, label={above right, inner sep=1pt}:{\tiny$i_2$}]{} to 
	(1,-1) node[V, label={below right, inner sep=1pt}:{\tiny$i_3$}]{} to 
	 (0,-1) node[V, label={below left, inner sep=1pt}:{\tiny$i_4$}]{} to (0,0);
\draw[densely dotted, thick, blue] (.5,.2) to (.5,-1.2) node[below]{\tiny $s$};
\draw[<-, red, bend left=50] (1.5,-.1) to node[right]{\tiny$r$} (1.5,-.9);
}
$$
(where $i_1, i_2, i_3, i_4 \in \{1,2\}$). For example, $r$ fixes $v_1 \otimes v_1 \otimes v_1 \otimes v_1$, but  $r \cdot v_1 \otimes v_2 \otimes v_1\otimes v_1 =  v_2 \otimes v_1 \otimes v_1\otimes v_1$.


Use the primitive central idempotents of $\CC D_8$ to decompose $V^{\otimes 4}$ into its isotypic components (you computed these idempotents in HW 5; you should also know which corresponds to which simple representations of $\CC D_8$). Then make a dimension argument to classify the decomposition of $V^{\otimes 4}$ up to isomorphism---and make a complete decomposition if you can. \\ $\quad$ \hfill {[\emph{See p.\ 2 for some help.}]}

\begin{proof}
We have that $M = V \otimes V \otimes V \otimes V = \CC\{v_{i_1} \otimes v_{i_2} \otimes v_{i_3} \otimes v_{i_4} \ | \ v_{i_j} \text{ is a basis vector of } V\}$. We can think about each element of $V$ as a length 4 word with alphabet $\{1,2\}$. For example, we have $v_1 \otimes v_1 \otimes v_2 \otimes v_1 \mapsto 1121$. In addition, we have $\dim(M) = \dim(V)^4 = 16$.
\end{proof}

\item Let $S_3 \le S_4$ in the usual way, and let $\cW$ be the reflection representation. Compute  the action of $s_1 = (12)$, $s_2 = (23)$, and $s_3 = (34)$ on $\Ind_{\CC S_3}^{\CC S_4}(\cW)$. \hfill {[\emph{Hint:} Stay organized!]}
\end{enumerate}

\begin{proof}
We have that $S_3 = \{1, (12), (23), (13), (123), (132)\}$ and $S_4 = \{1, (ab), (abc), (abcd), (ab)(cd) \ | \ a,b,c,d \in \{1,2,3,4\} \text{ distinct}\}$. By Lagrange's theorem, we have that $|S_4 : S_3| = |S_4|/|S_3| = 4!/3! = 4$. Thus, there should be $4$ left costs of $S_3$ in $S_4$. We will compute these cosets using the transpositions: $1, (14), (24), (34)$. This gives us,
\begin{align*}
1\{1, (12), (23), (13), (123), (132)\} &= \{1, (12), (23), (13), (123), (132)\}\\
(14)\{1, (12), (23), (13), (123), (132)\} &= \{(14), (124), (14)(23), (134), (1234), (1324)\}\\
(24)\{1, (12), (23), (13), (123), (132)\} &= \{(24), (142), (234), (24)(13), (1423), (1342)\}\\
(34)\{1, (12), (23), (13), (123), (132)\} &= \{(34), (34)(12), (243), (143), (1243), (1432)\}
\end{align*}

So $a_1 = 1, a_2 = (14), a_3 = (24), a_4 = (34)$.\\

Observe that $S_4$ is generated by $\langle (1234), (12) \rangle$, and so for each $a_i$, we will compute $(1234)a_i = a_j \sigma$ and $(12)a_i = a_k \tau$ for some $j,k \in \{1,2,3,4\}$ and some $\sigma, \tau \in S_3$:
\begin{align*}
(1234)a_1 &= (1234) \cdot 1 = 1 \cdot (1234) = a_1 \cdot (1234)\\
(1234)a_2 &= (1234) \cdot (14) = (234) = a_3 \cdot (23)\\
(1234)a_3 &= (1234) \cdot (24) = (21)(34) = (34)(12) = a_4 \cdot (12)\\
(1234)a_4 &= (1234) \cdot (34) = (312) = (123) = a_1 \cdot (123)\\
(12)a_1 &= (12) \cdot 1 = 1 \cdot (12) = a_1 \cdot (12)\\
(12)a_2 &= (12) \cdot (14) = (142) = (24) \cdot (12) = a_3 \cdot (12)\\
(12)a_3 &= (12) \cdot (24) = (241) = (124) = (14) \cdot (12) = a_2 \cdot (12)\\
(12)a_4 &= (12) \cdot (34) = (12)(34) = (34)(12) = a_4 \cdot (12)
\end{align*}

Now let us fix any $0 \neq \alpha \in \mathbb{C}$ as our basis. Then $\CC S_4 \otimes_{\CC S_3}$ has basis,
\begin{align*}
v_1 = a_1 \otimes \alpha, \; \; v_2 = a_2 \otimes \alpha, \; \; v_3 = a_3 \otimes \alpha, \; \; v_4 = a_4 \otimes \alpha
\end{align*}

By the previous computations, on this basis we have,
\begin{align*}
(12)v_1 = (12)a_1 \otimes \alpha &= a_1 \otimes (12)\alpha\\
&= a_1 \otimes (-\alpha) = -a_1 \otimes \alpha = -v_1
\end{align*}

\begin{align*}
(12)v_2 = (12)a_2 \otimes \alpha &= a_3 \otimes (12)\alpha\\
&= a_3 \otimes (-\alpha) = -a_1 \otimes \alpha = -v_3
\end{align*}

\begin{align*}
(12)v_3 = (12)a_3 \otimes \alpha &= a_2 \otimes (12)\alpha\\
&= a_2 \otimes (-\alpha) = -a_2 \otimes \alpha = -v_2
\end{align*}

\begin{align*}
(12)v_4 = (12)a_4 \otimes \alpha &= a_4 \otimes (12)\alpha\\
&= a_4 \otimes (-\alpha) = -a_4 \otimes \alpha = -v_4
\end{align*}

So $(12)v_1 = -v_1$, $(12)v_2 = -v_3$, $(12)v_3 = -v_2$, and $(12)v_4 = -v_4$. This yields,
\begin{align*}
\rho((12)) = \begin{pmatrix}
-1 & 0 & 0 & 0\\
0 & 0 & -1 & 0\\
0 & -1 & 0 & 0\\
0 & 0 & 0 & -1
\end{pmatrix}
\end{align*}

Now we will compute the action of $(1234)$ on 
\end{proof}

\vfill


\hrule
\emph{\small To receive credit for this assignment, include the following in your solutions [edited appropriately]:}

\smallskip

\textbf{Academic integrity statement:} I \emph{did not violate} the CUNY Academic Integrity Policy in completing this assignment. \hfill \emph{Christopher Hayduk}

\medskip
\hrule

\vfill

\pagebreak

\emph{Help with \#3:} This is a big computational problem. But with a little bit of care, it won't be too bad. One tip is to encode a basis vector like $v_{i_1} \otimes v_{i_2} \otimes v_{i_3} \otimes v_{i_4}$ as $i_1i_2i_3i_4$. For example, $v_1 \otimes v_2 \otimes v_1\otimes v_1$ becomes $1211$, and $r \cdot 1211 = 2111$. Another trick you have up your sleeve is action graphs; namely, the action of $\CC D_8$ on $V^{\otimes 4}$ is a linear extension of the action of $D_8$ on $\{i_1i_2i_3i_4 ~|~ i_\ell \in \{1,2\}\}$. For example, one part of your action graph will look like 
$$\TikZ{
	\node (1112) at (0,2) {1112}; 
	\node (1121) at (2,2) {1121}; 
	\node (1211) at (2,0) {1211}; 
	\node (2111) at (0,0) {2111}; 
	\draw[->] (1112) to node[above]{$r$} (1121);
	\draw[->] (1121) to node[right]{$r$} (1211);
	\draw[->] (1211) to node[below]{$r$} (2111);
	\draw[->] (2111) to node[left]{$r$} (1112);
	\draw[<->, bend right] (1112) to node[below]{$s$} (1121);
	\draw[<->, bend left] (2111) to node[above]{$s$} (1211);
}.$$
Next, your job is to compute $z_j V^{\otimes 4}$ for each $j = 1, \dots, 5$. But since the simple tensors $\{i_1i_2i_3i_4 ~|~ i_\ell \in \{1,2\}\}$ form a spanning set of $V^{\otimes 4}$; the action of $z_j$ on this set, $\{z_j \cdot i_1i_2i_3i_4 ~|~ i_\ell \in \{1,2\}\}$, sill form a spanning set of  $z_j V^{\otimes 4}$. To compute $z_j V^{\otimes 4}$, you just need to compute $z_j \cdot i_1i_2i_3i_4$ for each set of $i_\ell \in \{1,2\}$, and taking the span of the result. 


Now, recall that the coefficients in $z_1$ correspond to setting $r=1$ and $s=1$; the coefficients in $z_2$ correspond to setting $r=-1$ and $s=1$;  and so on\dots; so the first four of these computations essentially amount to walking around the vertices of this graph, assigning $\pm 1$ coefficients by what edge we walk along, and then summing up the result. So for example, the computation of $z_2$ acting on $1112$ looks like (starting from the upper-left corner, corresponding to the action of $1$, and moving out)

\centerline{$\TikZ{
	\node[thick, dgreen, rounded corners, draw, label={below right, inner sep=1pt, dgreen}:{\tiny$+$}] 
			(1112) at (0,0) {\color{black}1112}; 
	\node[thick, purple, rounded corners, draw, label={below right, inner sep=1pt, purple}:{\tiny$-$}]
			(1121) at (2,0) {\color{black}1121}; 
	\node[thick, dgreen, rounded corners, draw, label={below right, inner sep=1pt, dgreen}:{\tiny$+$}]
			(1211) at (4,0) {\color{black}1211}; 
	\node[thick, purple, rounded corners, draw, label={below right, inner sep=1pt, purple}:{\tiny$-$}]
			(2111) at (6,0) {\color{black}2111};  
	\node[thick, dgreen, rounded corners, draw, label={above right, inner sep=1pt, dgreen}:{\tiny$+$}]
			(1121b) at (0,-2) {\color{black}1121}; 
	\node[thick, purple, rounded corners, draw, label={above right, inner sep=1pt, purple}:{\tiny$-$}]
			(1211b) at (2,-2) {\color{black}1211}; 
	\node[thick, dgreen, rounded corners, draw, label={above right, inner sep=1pt, dgreen}:{\tiny$+$}]
			(2111b) at (4,-2) {\color{black}2111}; 
	\node[thick, purple, rounded corners, draw, label={above right, inner sep=1pt, purple}:{\tiny$-$}]
			(1112b) at (6,-2) {\color{black}1112};
	\node[black] (1) at (0,.5) {$1$}; 
	\node[black] (r) at (2,.5) {$r$}; 
	\node[black] (rr) at (4,.5) {$r^2$}; 
	\node[black] (rrr) at (6,.5) {$r^3$};  
	\node[black] (s) at (0,-2.5) {$s$}; 
	\node[black] (rs) at (2,-2.5) {$rs$}; 
	\node[black] (rrs) at (4,-2.5) {$r^2s$}; 
	\node[black] (rrrs) at (6,-2.5) {$r^3s$};
	\draw[thick, red, ->] (1112) to node[above]{\tiny$r$} (1121);
	\draw[thick, red, ->] (1121) to node[above]{\tiny$r$} (1211);
	\draw[thick, red, ->] (1211) to node[above]{\tiny$r$} (2111);
	\draw[thick, blue, ->, bend right] (1112) to node[left]{\tiny$s$} (1121b);
	\draw[thick, red, ->] (1121b) to node[below]{\tiny$r$} (1211b);
	\draw[thick, red, ->] (1211b) to node[below]{\tiny$r$} (2111b);
	\draw[thick, red, ->] (2111b) to node[below]{\tiny$r$} (1112b);
	\node[inner sep = 2pt] (s) at (-1,.5) {\tiny start}; \draw [->, bend right] (s) to (1112);
} \quad\begin{array}{l} \text{so } z_2 \cdot 1112 = \frac{1}{8}\left(\gV{1112} - \pV{1121}+\gV{1211}-\pV{2111}\right.\\\hspace{1.2in}\left.+\gV{1121}-\pV{1211}+\gV{2111}-\pV{1112}\right) = 0.)\end{array}$}



Continue computing the actions of the $z_j$ on the basis vectors, organize your computations by orbits. 
For example, setting 
$$b_1 = 1112, \quad b_2 = 1121, \quad b_3 = 1211, \quad \text{and} \quad  b_4 = 2111,$$
we have 
\begin{align*}
z_1 b_i &= \frac{1}{4}(b_1 + b_2 + b_3 + b_4) & \text{ for }i=1,2,3,4;\\
z_2 b_i &= 0 \quad \text{ and } \quad z_4 b_i = 0& \text{ for }i=1,2,3,4;\\
z_3 b_1& = z_3 b_3 = -z_3 b_2 = -z_3 b_4 = \frac{1}{4}(b_1 - b_2 + b_3 - b_4); &\\
z_5 b_1& = -z_5 b_3 = \frac{1}{2}(b_1 - b_3); \quad \text{ and } \quad
z_5 b_2 = -z_5 b_4 = \frac{1}{2}(b_2 - b_4).
\end{align*}
So 
\begin{align*}
z_1V^{\otimes 4} & \text{ contains } b_1 + b_2 + b_3 + b_4;\\
z_3V^{\otimes 4} & \text{ contains } b_1 - b_2 + b_3 - b_4; \quad \text{ and}\\
z_5V^{\otimes 4}  & \text{ contains } b_1 - b_3 \text{ and } b_2 - b_4.
\end{align*} 
(We have accounted for 4 of 16 dim'ns in $V^{\otimes 4}$, so we're now 1/4 done with this computation!)

\end{document}