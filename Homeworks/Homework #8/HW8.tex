\documentclass[11pt, reqno]{amsart}
\usepackage[margin=1in]{geometry}    
\geometry{letterpaper}       
%\geometry{landscape}                % Activate for for rotated page geometry
\usepackage[parfill]{parskip}    % Activate to begin paragraphs with an empty line rather than an indent
\usepackage{amsfonts, amscd, amssymb, amsthm, amsmath}
\usepackage{pdfsync}  %leaves makers for tex searching
\usepackage{enumerate}
\usepackage[pdftex,bookmarks]{hyperref}



%%% Theorems %%%--------------------------------------------------------- 
\theoremstyle{plain}
	\newtheorem{thm}{Theorem}[section]
	\newtheorem{lemma}[thm]{Lemma}
	\newtheorem{prop}[thm]{Proposition}
	\newtheorem{cor}[thm]{Corollary}
\theoremstyle{definition}
	\newtheorem*{defn}{Definition}
	\newtheorem{remark}[thm]{Remark}
\theoremstyle{example}
	\newtheorem*{example}{Example}


%%% Environments %%%--------------------------------------------------------- 
\newenvironment{ans}{\medskip \paragraph*{\emph{Answer}.}}{\hfill \break  $~\!\!$ \dotfill \medskip }
\newenvironment{sketch}{\medskip \paragraph*{\emph{Proof sketch}.}}{ \medskip }
\newenvironment{summary}{\medskip \paragraph*{\emph{Summary}.}}{  \hfill \break  \rule{1.5cm}{0.4pt} \medskip }
\newcommand\Ans[1]{\hfill \emph{Answer:} {#1}}


%%% Pictures %%%--------------------------------------------------------- 
%%% If you need to draw pictures, tikzpicture is one good option. Here are some basic things I always use:
\usepackage{tikz}
\usetikzlibrary{arrows}
\usetikzlibrary{shapes}
\tikzstyle{V}=[draw, fill =black, circle, inner sep=0pt, minimum size=2pt]
\newcommand\TikZ[1]{\begin{matrix}\begin{tikzpicture}#1\end{tikzpicture}\end{matrix}}





%%% Color  %%%---------------------------------------------------------
\usepackage{color}
\newcommand{\NOTE}[1]{{\color{blue}#1}}
\newcommand{\blue}[1]{{\color{blue}#1}}
\newcommand{\red}[1]{{\color{red}#1}}
\newcommand{\MOVED}[1]{{\color{gray}#1}}


%%% Alphabets %%%---------------------------------------------------------
%%% Some shortcuts for my commonly used special alphabets and characters.
\def\cA{\mathcal{A}}\def\cB{\mathcal{B}}\def\cC{\mathcal{C}}\def\cD{\mathcal{D}}\def\cE{\mathcal{E}}\def\cF{\mathcal{F}}\def\cG{\mathcal{G}}\def\cH{\mathcal{H}}\def\cI{\mathcal{I}}\def\cJ{\mathcal{J}}\def\cK{\mathcal{K}}\def\cL{\mathcal{L}}\def\cM{\mathcal{M}}\def\cN{\mathcal{N}}\def\cO{\mathcal{O}}\def\cP{\mathcal{P}}\def\cQ{\mathcal{Q}}\def\cR{\mathcal{R}}\def\cS{\mathcal{S}}\def\cT{\mathcal{T}}\def\cU{\mathcal{U}}\def\cV{\mathcal{V}}\def\cW{\mathcal{W}}\def\cX{\mathcal{X}}\def\cY{\mathcal{Y}}\def\cZ{\mathcal{Z}}

\def\AA{\mathbb{A}} \def\BB{\mathbb{B}} \def\CC{\mathbb{C}} \def\DD{\mathbb{D}} \def\EE{\mathbb{E}} \def\FF{\mathbb{F}} \def\GG{\mathbb{G}} \def\HH{\mathbb{H}} \def\II{\mathbb{I}} \def\JJ{\mathbb{J}} \def\KK{\mathbb{K}} \def\LL{\mathbb{L}} \def\MM{\mathbb{M}} \def\NN{\mathbb{N}} \def\OO{\mathbb{O}} \def\PP{\mathbb{P}} \def\QQ{\mathbb{Q}} \def\RR{\mathbb{R}} \def\SS{\mathbb{S}} \def\TT{\mathbb{T}} \def\UU{\mathbb{U}} \def\VV{\mathbb{V}} \def\WW{\mathbb{W}} \def\XX{\mathbb{X}} \def\YY{\mathbb{Y}} \def\ZZ{\mathbb{Z}}  

\def\fa{\mathfrak{a}} \def\fb{\mathfrak{b}} \def\fc{\mathfrak{c}} \def\fd{\mathfrak{d}} \def\fe{\mathfrak{e}} \def\ff{\mathfrak{f}} \def\fg{\mathfrak{g}} \def\fh{\mathfrak{h}} \def\fj{\mathfrak{j}} \def\fk{\mathfrak{k}} \def\fl{\mathfrak{l}} \def\fm{\mathfrak{m}} \def\fn{\mathfrak{n}} \def\fo{\mathfrak{o}} \def\fp{\mathfrak{p}} \def\fq{\mathfrak{q}} \def\fr{\mathfrak{r}} \def\fs{\mathfrak{s}} \def\ft{\mathfrak{t}} \def\fu{\mathfrak{u}} \def\fv{\mathfrak{v}} \def\fw{\mathfrak{w}} \def\fx{\mathfrak{x}} \def\fy{\mathfrak{y}} \def\fz{\mathfrak{z}}
\def\fgl{\mathfrak{gl}}  \def\fsl{\mathfrak{sl}}  \def\fso{\mathfrak{so}}  \def\fsp{\mathfrak{sp}}  
\def\GL{\mathrm{GL}} \def\SL{\mathrm{SL}}  \def\SP{\mathrm{SL}}

\def\<{\langle} \def\>{\rangle}
\def\({\<\!\<}\def\){\>\!\>}
\def\ad{\mathrm{ad}} 
\def\Aut{\mathrm{Aut}}
\def\dim{\mathrm{dim}} 
\def\End{\mathrm{End}} 
\def\ev{\mathrm{ev}} 
\def\half{\hbox{$\frac12$}}
\def\img{\mathrm{img}}
\def\Hom{\mathrm{Hom}} 
\def\Fn{\mathrm{Fn}} 
\def\Fr{\mathcal{F}\mathrm{r}}
\def\hgt{\mathrm{ht}} 
\def\id{\mathrm{id}} 
\def\qtr{\mathrm{qtr}} 
\def\sgn{\mathrm{sgn}}
\def\supp{\mathrm{supp}}
\def\tr{\mathrm{tr}} 
\def\Tor{\mathrm{Tor}} 
\def\vep{\varepsilon}
\def\f{\varphi}



\def\Obj{\mathrm{Obj}}
\def\normeq{\unlhd}
\def\Set{{\cS\mathrm{et}}}
\def\Fin{{\cF\mathrm{inSet}}}
\def\Set{{\cS\mathrm{et}}}
\def\Grp{{\cG\mathrm{rp}}}
\def\Ab{{\cA\mathrm{b}}}
\def\Mod{{\cM\mathrm{od}}}
\def\ab{\mathrm{ab}}

% Arrows:
\newcommand\xdhrightarrow[2][]{%
  \mathrel{\ooalign{$\xrightarrow[#1\mkern4mu]{#2\mkern4mu}$\cr%
  \hidewidth$\rightarrow\mkern4mu$}}
}
%\newcommand\dhrightarrow{%
%  \mathrel{\ooalign{$\rightarrow$\cr%
%  $\mkern3.5mu\rightarrow$}}
%}
\def\dhrightarrow{\twoheadrightarrow}
\def\dhleftarrow{\twoheadleftarrow}


%%%%%%%%%%%%%%%%%%%%%%%%%%%%%% 
%%%%%%%%%%%%%%%%%%%%%%%%%%%%%%

\def\HW{8}
\def\DUE{4/23/2021}

\title[Homework \HW]{Homework \HW \\
Math B4900\\
\small Due: \DUE}
\author{}
%\date{}                                           % Activate to display a given date or no date

\begin{document}
%\maketitle %%% COMMENT THIS OUT and UNCOMMENT the following to give yourself a good assignment header:
\begin{flushright}
Chris Hayduk\\
Math B4900\\
Homework \HW\\
\DUE
\end{flushright}

Let $A$ be a ring with 1.
\begin{enumerate}[1.]
\item Prove that if every $A$-module is free, then $A$ is a division ring. {[\emph{Caution:} this proof may involve a lot of machinery. Give it time, and ample brainstorming of the tools you have so far.]}
\begin{proof}
Note that $A$ is a division ring if $A^x = A - \{0\}$. That is, every non-zero element is a unit.\\

Suppose $A$ is a ring with $1$ and that every $A$-module is free. Properties of $A$: every $A$-module is projective, every $A$-module is injective, $A$ is semisimple. Since $A$ is semisimple, then the left regular $A$-module is completely decomposible (semisimple). In addition, $A \cong M_{n_1}(\Delta_1) \times \cdots \times M_{n_{\ell}}(\Delta_{\ell})$.\\

Now we will show that $A$ is simple as a left regular $A$-module in order to show that every non-zero element of $A$ has a multiplicative inverse.\\

We have that $A$ is a submodule of $A$. Then we must have that $A/A$ is semisimple. But $A/A = 0_A$.

\end{proof}
\item Classify the semisimple $\ZZ$-modules. {[\emph{Hint.} What are the simple $\ZZ$-modules?]}

\begin{proof}
Note that a $\ZZ$ module is simple if it is of the form $\ZZ/I$ where $I$ is a maximal ideal. The ideals of $\ZZ$ are precisely the sets of all integers divisible by a fixed integer $n$. That is, $n\ZZ$ is an ideal for all $n \in \ZZ$. Recall that an ideal $n\ZZ$ of $\ZZ$ is maximal if there are no other ideals of the form $k\ZZ$ such that $n\ZZ \subset k\ZZ \subset \ZZ$. Observe that if $n$ is a composite integer, then we can write $n = p_1p_2 \cdots p_{\ell}$ for primes in $\ZZ$. That is, for any $p_j$ in that expansion, we have that $p_j$ divides $n$ and thus all multiples of $n$. Hence, $n\ZZ \subset p_j \ZZ$ for any prime $p_j$ in that expansion. Moreover, for every prime we must have that there is no integer $m$ such that $p_j \ZZ \subset m \ZZ$, otherwise $m$ would divide $p_j$ and hence $p_j$ would not be prime. Thus, the maximal ideals of $\ZZ$ are precisely of the form $p\ZZ$ where $p$ is a prime.\\

Now we have that the simple modules of $\ZZ$ are of the form $\ZZ/p\ZZ$ for all primes $p \in \ZZ$. Since semisimple modules are direct sums of simple modules, we have that any semisimple module of $\ZZ$ is of the form:
\begin{align*}
p_1 \ZZ \oplus p_2 \ZZ \oplus \cdots \oplus p_{\ell} \ZZ
\end{align*}

for some primes $p_1, \ldots, p_{\ell}$ (not necessarily distinct).
\end{proof}


\item Let $M$ be a semisimple $A$-module. Prove that the following are equivalent: 
\begin{enumerate}[(i)]
\item $M$ is finitely-generated;
\item $M$ is Noetherian;
\item $M$ is Artinian;
\item $M$ is a finite direct sum of simple modules. 
\end{enumerate}

\begin{proof}
First we will show that (i) is equivalent to (ii). Suppose $M$ is finitely generated. Then there exist $m_1, m_2, \ldots, m_n \in M$ such that for any $x \in M$, there exist $a_1, a_2, \ldots, a_n \in A$ with $x = a_1m_1 + a_2m_2 + \cdots + a_nm_n$. Since every element of a submodule of $M$ is also an element of $M$, then it must be true that every element of a submodule $N$ of $M$ is finitely generated as well. Hence, $M$ is Noetherian. Now suppose $M$ is Noetherian. Then every submodule of $M$ is finitely generated. In particular, since $M$ is a submodule of itself, it must be finitely generated. Thus, (i) and (ii) are equivalent.\\

Now we will show the equivalence of (i) and (iv). Suppose $M$ is semisimple and finitely generated. Then $M$ is the direct sum of simple modules and, since (i) is equivalent to (ii), each of those submodules is finitely generated. Since the generators of $M$ are finite, they can old by combined in a finite number of ways. Hence, there must be finitely many submodules which are finitely generated. Hence, $M$ is a finite direct sum of simple modules. Now let us assume that $M$ is a finite direct sum of simple modules and work towards the other directions. Every simple module is cyclic and hence generated by one element. The union of these generators forms a basis for $M$ since $M$ is a direct sum of these simple modules. Since there are a finite number of these simple modules, then $M$ is finitely-generated by this union as required. Hence, by this and our previous work, (i), (ii), and (iv) are equivalent.\\

Now we will show the equivalent of (iii) and (iv). Suppose $M$ is semisimple and Artinian. Then the sequence of submodules of $M$
\begin{align*}
M_1 \supset M_2 \supset \ldots
\end{align*}
stabilizes. That is, there exists an integer $N$ such that if $n \geq N$ then $M_n = M_{n+1}$. Since no simple module can have a submodule, then $M_N$ is the only simple module in this chain. Observe that since $M$ is semisimple, it must be the direct sum of simple submodules. There must be only finitely many of these simple submodules (why?), so $M$ is a finite direct sum of simple modules. Now let us assume that $M$ is semisimple and a finite direct sum of simple modules and work in reverse.
\end{proof}

\item 
\begin{enumerate}[(a)]
\item Let $R$ and $S$ be rings such that $M_m(R) \cong M_n(S)$ for some $m, n \in \ZZ_{\ge 1}$. Does this imply that $m = n$ and $R \cong S$? If so, why? If not, give a counter-example. 

\begin{proof}
This is not true. Let us take $R = M_k(\CC)$ and $S = M_n(\CC)$ with $n \neq k$ and $n, k \geq 1$. Then $S \not\cong R$ since the matrices are of different dimension, but we have that,
\begin{align*}
M_n(R) = M_n(M_k(\CC)) \cong M_{nk}(\CC)
\end{align*}

and
\begin{align*}
M_k(S) = M_k(M_n(\CC)) \cong M_{nk}(\CC)
\end{align*}

Hence, since $M_n(R) \cong M_{nk}(\CC)$ and $M_k(S) \cong M_{nk}(\CC)$, we must have that $M_n(R) \cong M_k(S)$.
\end{proof}

\item We call $A$ a \emph{full matrix ring} if $A \cong M_n(R)$ for some ring $R$ and some $n \in \ZZ_{\ge 1}$. Is the homomorphic image of a matrix ring necessarily itself a matrix ring? If so, prove it. If not, give a counterexample.  

\begin{proof}
Let $B$ be the homomorphic image of $A$. By The First Isomorphism Theorem from Dummit and Foote, we have that $B \cong A/\ker \phi$. In addition, from Theroem 7(2) in Dummit and Foote, we have that $\ker \phi \cong I$ where $I$ is some ideal of $A$. Hence, $B \cong A/I \cong M_n(R)/I$. Moreover, we know from Lam Theorem 3.1 that the ideals of $M_n(R)$ are in bijection with the ideals of $R$. That is, for some ideal $I_R$ of $R$, we have,
\begin{align*}
B &\cong M_n(R)/I\\
&\cong M_n(R)/M_n(I_R)\\
&\cong M_n(R/I_R)
\end{align*}

Hence, $B$ is a matrix ring as well.
\end{proof}

\end{enumerate}
\end{enumerate}

\vfill


\hrule
\emph{\small To receive credit for this assignment, include the following in your solutions [edited appropriately]:}

\smallskip

\textbf{Academic integrity statement:} I \emph{did not violate} the CUNY Academic Integrity Policy in completing this assignment. \hfill \emph{Chris Hayduk}

\medskip
\hrule

\vfill


\end{document}