\documentclass[11pt, reqno]{amsart}
\usepackage[margin=1in]{geometry}    
\geometry{letterpaper}       
%\geometry{landscape}                % Activate for for rotated page geometry
\usepackage[parfill]{parskip}    % Activate to begin paragraphs with an empty line rather than an indent
\usepackage{amsfonts, amscd, amssymb, amsthm, amsmath}
\usepackage{pdfsync}  %leaves makers for tex searching
\usepackage{enumerate}
\usepackage[pdftex,bookmarks]{hyperref}



%%% Theorems %%%--------------------------------------------------------- 
\theoremstyle{plain}
	\newtheorem{thm}{Theorem}[section]
	\newtheorem{lemma}[thm]{Lemma}
	\newtheorem{prop}[thm]{Proposition}
	\newtheorem{cor}[thm]{Corollary}
\theoremstyle{definition}
	\newtheorem*{defn}{Definition}
	\newtheorem{remark}[thm]{Remark}
\theoremstyle{example}
	\newtheorem*{example}{Example}


%%% Environments %%%--------------------------------------------------------- 
\newenvironment{ans}{\medskip \paragraph*{\emph{Answer}.}}{\hfill \break  $~\!\!$ \dotfill \medskip }
\newenvironment{sketch}{\medskip \paragraph*{\emph{Proof sketch}.}}{ \medskip }
\newenvironment{summary}{\medskip \paragraph*{\emph{Summary}.}}{  \hfill \break  \rule{1.5cm}{0.4pt} \medskip }
\newcommand\Ans[1]{\hfill \emph{Answer:} {#1}}


%%% Pictures %%%--------------------------------------------------------- 
%%% If you need to draw pictures, tikzpicture is one good option. Here are some basic things I always use:
\usepackage{tikz}
\usetikzlibrary{arrows}
\usetikzlibrary{shapes}
\tikzstyle{V}=[draw, fill =black, circle, inner sep=0pt, minimum size=2pt]
\newcommand\TikZ[1]{\begin{matrix}\begin{tikzpicture}#1\end{tikzpicture}\end{matrix}}





%%% Color  %%%---------------------------------------------------------
\usepackage{color}
\newcommand{\NOTE}[1]{{\color{blue}#1}}
\newcommand{\blue}[1]{{\color{blue}#1}}
\newcommand{\red}[1]{{\color{red}#1}}
\newcommand{\MOVED}[1]{{\color{gray}#1}}


%%% Alphabets %%%---------------------------------------------------------
%%% Some shortcuts for my commonly used special alphabets and characters.
\def\cA{\mathcal{A}}\def\cB{\mathcal{B}}\def\cC{\mathcal{C}}\def\cD{\mathcal{D}}\def\cE{\mathcal{E}}\def\cF{\mathcal{F}}\def\cG{\mathcal{G}}\def\cH{\mathcal{H}}\def\cI{\mathcal{I}}\def\cJ{\mathcal{J}}\def\cK{\mathcal{K}}\def\cL{\mathcal{L}}\def\cM{\mathcal{M}}\def\cN{\mathcal{N}}\def\cO{\mathcal{O}}\def\cP{\mathcal{P}}\def\cQ{\mathcal{Q}}\def\cR{\mathcal{R}}\def\cS{\mathcal{S}}\def\cT{\mathcal{T}}\def\cU{\mathcal{U}}\def\cV{\mathcal{V}}\def\cW{\mathcal{W}}\def\cX{\mathcal{X}}\def\cY{\mathcal{Y}}\def\cZ{\mathcal{Z}}

\def\AA{\mathbb{A}} \def\BB{\mathbb{B}} \def\CC{\mathbb{C}} \def\DD{\mathbb{D}} \def\EE{\mathbb{E}} \def\FF{\mathbb{F}} \def\GG{\mathbb{G}} \def\HH{\mathbb{H}} \def\II{\mathbb{I}} \def\JJ{\mathbb{J}} \def\KK{\mathbb{K}} \def\LL{\mathbb{L}} \def\MM{\mathbb{M}} \def\NN{\mathbb{N}} \def\OO{\mathbb{O}} \def\PP{\mathbb{P}} \def\QQ{\mathbb{Q}} \def\RR{\mathbb{R}} \def\SS{\mathbb{S}} \def\TT{\mathbb{T}} \def\UU{\mathbb{U}} \def\VV{\mathbb{V}} \def\WW{\mathbb{W}} \def\XX{\mathbb{X}} \def\YY{\mathbb{Y}} \def\ZZ{\mathbb{Z}}  

\def\fa{\mathfrak{a}} \def\fb{\mathfrak{b}} \def\fc{\mathfrak{c}} \def\fd{\mathfrak{d}} \def\fe{\mathfrak{e}} \def\ff{\mathfrak{f}} \def\fg{\mathfrak{g}} \def\fh{\mathfrak{h}} \def\fj{\mathfrak{j}} \def\fk{\mathfrak{k}} \def\fl{\mathfrak{l}} \def\fm{\mathfrak{m}} \def\fn{\mathfrak{n}} \def\fo{\mathfrak{o}} \def\fp{\mathfrak{p}} \def\fq{\mathfrak{q}} \def\fr{\mathfrak{r}} \def\fs{\mathfrak{s}} \def\ft{\mathfrak{t}} \def\fu{\mathfrak{u}} \def\fv{\mathfrak{v}} \def\fw{\mathfrak{w}} \def\fx{\mathfrak{x}} \def\fy{\mathfrak{y}} \def\fz{\mathfrak{z}}
\def\fgl{\mathfrak{gl}}  \def\fsl{\mathfrak{sl}}  \def\fso{\mathfrak{so}}  \def\fsp{\mathfrak{sp}}  
\def\GL{\mathrm{GL}} \def\SL{\mathrm{SL}}  \def\SP{\mathrm{SL}}

\def\<{\langle} \def\>{\rangle}
\def\({\<\!\<}\def\){\>\!\>}
\def\ad{\mathrm{ad}} 
\def\Aut{\mathrm{Aut}}
\def\dim{\mathrm{dim}} 
\def\End{\mathrm{End}} 
\def\ev{\mathrm{ev}} 
\def\half{\hbox{$\frac12$}}
\def\img{\mathrm{img}}
\def\Hom{\mathrm{Hom}} 
\def\Fn{\mathrm{Fn}} 
\def\hgt{\mathrm{ht}} 
\def\id{\mathrm{id}} 
\def\qtr{\mathrm{qtr}} 
\def\tr{\mathrm{tr}} 
\def\Tor{\mathrm{Tor}} 
\def\sgn{\mathrm{sgn}}
\def\vep{\varepsilon}
\def\f{\varphi}

% Arrows:
\newcommand\xdhrightarrow[2][]{%
  \mathrel{\ooalign{$\xrightarrow[#1\mkern4mu]{#2\mkern4mu}$\cr%
  \hidewidth$\rightarrow\mkern4mu$}}
}
%\newcommand\dhrightarrow{%
%  \mathrel{\ooalign{$\rightarrow$\cr%
%  $\mkern3.5mu\rightarrow$}}
%}
\def\dhrightarrow{\twoheadrightarrow}
\def\dhleftarrow{\twoheadleftarrow}


%%%%%%%%%%%%%%%%%%%%%%%%%%%%%% 
%%%%%%%%%%%%%%%%%%%%%%%%%%%%%%

\def\HW{4}
\def\DUE{3/14/2021}

\title[Homework \HW]{Homework \HW \\
Math B4900\\
\small Due: \DUE}
\author{}
%\date{}                                           % Activate to display a given date or no date

\begin{document}
%\maketitle %%% COMMENT THIS OUT and UNCOMMENT the following to give yourself a good assignment header:
\begin{flushright}
Chris Hayduk\\
Math B4900\\
Homework \HW\\
\DUE
\end{flushright}


Let $A$ be a ring with 1, and let $M$ be an $A$-module.
\begin{enumerate}[1.]
\item Let $z$ be a central element of $A$. Show that $zM$ is a submodule of $M$ and that $\f: M \to M$ defined by $m \mapsto zm$ is an $A$-module endomorphism.

\item Let $I$ be an ideal of $A$. Show that 
$$IM = \left\{ \left. \sum_{\text{fin.}} \alpha m_\alpha ~\right|~ \alpha \in I, m_\alpha \in M\right\}$$
is a submodule of $M$. Give an example where $IM = M$ and an example where $0 \ne IM \subsetneq M$ Finally, give an example showing that if $B$ is a subring of $A$, then $BM$ is not necessarily a submodule of $M$.

\item Let $\f: M \to N$ be an $A$-module homomorphism. Prove that $\ker(\f)$ is a submodule of $M$ and $\img(\f)$ is a submodule of $N$. Further, show that if $M$ and $N$ are simple, then either $\f = 0$ or  $\f$ is an isomorphism. {[You may use anything that we have already proven for groups (since $M$ and $N$ are also additive groups.]}

\item Let $X$ and $Y$ be submodules of $M$. Show that
$$0 \hookrightarrow X \cap Y \xrightarrow{f: x \mapsto (x,x)} 
	X \oplus Y \xrightarrow{g: (x,y) \mapsto x + y} X + Y \to 0$$
is a short exact sequence of $A$-modules. 

\item Let 
$$0 \hookrightarrow X \xrightarrow{f} Y \xrightarrow{g} Z \to 0 \quad \text{and} \quad 
 	0 \hookrightarrow X' \xrightarrow{f'} Y' \xrightarrow{g'} Z' \to 0$$
	be short exact sequences. A collection of homomorphisms 
	$$\alpha: X \to X', \quad \beta: Y \to Y', \quad \text{ and } \gamma: Z \to Z'$$
is a \emph{homomorphism of exact sequences} if the following diagram commutes:
\begin{equation}\label{SEShom}
\TikZ{[yscale=1.5, xscale=3]
\node (0') at (.3, 1) {$0$};
\node (0R') at (3.7, 1) {$0$};
\node (X') at (1,1) {$X'$}; 
\node (Y') at (2,1) {$Y'$}; 
\node (Z') at (3,1) {$Z'$}; 
\node (0) at (.3, 2) {$0$};
\node (0R) at (3.7, 2) {$0$};
\node (X) at (1,2) {$X$}; 
\node (Y) at (2,2) {$Y$}; 
\node (Z) at (3,2) {$Z$}; 
\foreach \x/\y in {X/\alpha,Y/\beta,Z/\gamma} {\draw [->] (\x) to node[left]{$\y$}
%	node[sloped, above]{$\sim$} 
	(\x');}
\draw[right hook-latex] (0) to (X);
\draw[right hook-latex] (0') to (X');
\draw[->] (X) to node[above] {$f$} (Y);
\draw[->] (Y) to node[above] {$g$} (Z);
\draw[->] (X') to node[below] {$f'$} (Y');
\draw[->] (Y') to node[below] {$g$} (Z');
\draw[->] (Z) to (0R); \draw[->] (Z') to (0R');
}
\end{equation}
If $\alpha, \beta$, and $\gamma$ are isomorphisms, then this is an \emph{isomorphism} or \emph{equivalence of exact sequences}.

\pagebreak

\begin{enumerate}
\item Fill in the unknown modules and homomorphisms to make the following diagram into a homomorphism of short exact sequences

$$
\TikZ{[yscale=1.5, xscale=3]
\node (0') at (.3, 1) {$0$};
\node (0R') at (3.7, 1) {$0$};
\node[draw, circle] (X') at (1,1) {?}; 
\node (Y') at (2,1) {$\ZZ$}; 
\node (Z') at (3,1) {$\ZZ/2\ZZ$}; 
\node (0) at (.3, 2) {$0$};
\node (0R) at (3.7, 2) {$0$};
\node[draw, circle]  (X) at (1,2) {?}; 
\node (Y) at (2,2) {$\ZZ$}; 
\node(Z) at (3,2) {$\ZZ/6\ZZ$}; 
\draw [->] (X) to node[left, outer sep=3pt, draw, rounded corners] {?} (X');
\draw [->] (Y) to node[left]{$*3$}(Y');
\draw [->] (Z) to node[left, outer sep=3pt, draw, rounded corners] {?} (Z');
\draw[right hook-latex] (0) to (X);
\draw[right hook-latex] (0') to (X');
\draw[->] (X) to node [above, outer sep=3pt, draw, rounded corners] {?}  (Y);
\draw[->] (Y) to node[above] {$\pi$} (Z);
\draw[->] (X') to node[below, outer sep=3pt, draw, rounded corners] {?} (Y');
\draw[->] (Y') to node[below] {$\pi$} (Z');
\draw[->] (Z) to (0R); \draw[->] (Z') to (0R');
}
$$

\item If you have a homomorphism of short exact sequences as in \eqref{SEShom}, and you know something about the properties of $\alpha$, $\beta$, or $\gamma$, what else (if anything) can you infer about the other two? Namely, fill in the blank (briefly justifying your answers):

\begin{center}
\emph{If  ($\alpha/\beta/\gamma$) is (injective/surjective/bijective), then...}

{\def\arraystretch{1.3}\begin{tabular}{|c||c|c|c|}\hline
 & injective & surjective & bijective\\\hline\hline
 $\alpha$ &&&\\\hline
 $\beta$ && (e.g.\ \emph{$\gamma$ is surjective})&\\\hline
 $\gamma$ &&& \\\hline
\end{tabular}}
\end{center}

{[For example, if $\beta$ is surjective, then you can infer that $\gamma$ is surjective, but nothing can be said about $\alpha$. (Why?)
]}


\end{enumerate}
\end{enumerate}

\vfill


\hrule
\emph{\small To receive credit for this assignment, include the following in your solutions [edited appropriately]:}

\smallskip

\textbf{Academic integrity statement:} I \emph{[violated/did not violate]} the CUNY Academic Integrity Policy in completing this assignment. \hfill \emph{[enter your full name as a digital signature here]}

\medskip
\hrule

\vfill


\end{document}