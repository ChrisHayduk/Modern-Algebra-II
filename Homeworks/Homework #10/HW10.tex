\documentclass[11pt, reqno]{amsart}
\usepackage[margin=1in]{geometry}    
\geometry{letterpaper}       
%\geometry{landscape}                % Activate for for rotated page geometry
\usepackage[parfill]{parskip}    % Activate to begin paragraphs with an empty line rather than an indent
\usepackage{amsfonts, amscd, amssymb, amsthm, amsmath}
\usepackage{pdfsync}  %leaves makers for tex searching
\usepackage{enumerate}
\usepackage[pdftex,bookmarks]{hyperref}



%%% Theorems %%%--------------------------------------------------------- 
\theoremstyle{plain}
	\newtheorem{thm}{Theorem}[section]
	\newtheorem{lemma}[thm]{Lemma}
	\newtheorem{prop}[thm]{Proposition}
	\newtheorem{cor}[thm]{Corollary}
\theoremstyle{definition}
	\newtheorem*{defn}{Definition}
	\newtheorem{remark}[thm]{Remark}
\theoremstyle{example}
	\newtheorem*{example}{Example}


%%% Environments %%%--------------------------------------------------------- 
\newenvironment{ans}{\medskip \paragraph*{\emph{Answer}.}}{\hfill \break  $~\!\!$ \dotfill \medskip }
\newenvironment{sketch}{\medskip \paragraph*{\emph{Proof sketch}.}}{ \medskip }
\newenvironment{summary}{\medskip \paragraph*{\emph{Summary}.}}{  \hfill \break  \rule{1.5cm}{0.4pt} \medskip }
\newcommand\Ans[1]{\hfill \emph{Answer:} {#1}}


%%% Pictures %%%--------------------------------------------------------- 
%%% If you need to draw pictures, tikzpicture is one good option. Here are some basic things I always use:
\usepackage{tikz}
\usetikzlibrary{arrows}
\usetikzlibrary{shapes}
\tikzstyle{V}=[draw, fill =black, circle, inner sep=0pt, minimum size=2pt]
\tikzstyle{bV}=[draw, fill =black, circle, inner sep=0pt, minimum size=4pt]
\newcommand\TikZ[1]{\begin{matrix}\begin{tikzpicture}#1\end{tikzpicture}\end{matrix}}





%%% Color  %%%---------------------------------------------------------
\usepackage{color}
\newcommand{\NOTE}[1]{{\color{blue}#1}}
\newcommand{\blue}[1]{{\color{blue}#1}}
\newcommand{\red}[1]{{\color{red}#1}}
\newcommand{\MOVED}[1]{{\color{gray}#1}}
\definecolor{dred}{rgb}{.8,0,.1}
\definecolor{dgreen}{rgb}{0,.6,.1}
\definecolor{purple}{rgb}{.6,0,.8}
\definecolor{dorange}{rgb}{.8,.25,0}
\definecolor{dyellow}{rgb}{.95,.85,0}
\definecolor{alert}{rgb}{.8,.25,0}

\newcommand{\gV}[1]{\TikZ{\node[thick, dgreen, rounded corners, draw] at (0,0){#1};}}
\newcommand{\pV}[1]{\TikZ{\node[thick, purple, rounded corners, draw] at (0,0){#1};}}


%%% Alphabets %%%---------------------------------------------------------
%%% Some shortcuts for my commonly used special alphabets and characters.
\def\cA{\mathcal{A}}\def\cB{\mathcal{B}}\def\cC{\mathcal{C}}\def\cD{\mathcal{D}}\def\cE{\mathcal{E}}\def\cF{\mathcal{F}}\def\cG{\mathcal{G}}\def\cH{\mathcal{H}}\def\cI{\mathcal{I}}\def\cJ{\mathcal{J}}\def\cK{\mathcal{K}}\def\cL{\mathcal{L}}\def\cM{\mathcal{M}}\def\cN{\mathcal{N}}\def\cO{\mathcal{O}}\def\cP{\mathcal{P}}\def\cQ{\mathcal{Q}}\def\cR{\mathcal{R}}\def\cS{\mathcal{S}}\def\cT{\mathcal{T}}\def\cU{\mathcal{U}}\def\cV{\mathcal{V}}\def\cW{\mathcal{W}}\def\cX{\mathcal{X}}\def\cY{\mathcal{Y}}\def\cZ{\mathcal{Z}}

\def\AA{\mathbb{A}} \def\BB{\mathbb{B}} \def\CC{\mathbb{C}} \def\DD{\mathbb{D}} \def\EE{\mathbb{E}} \def\FF{\mathbb{F}} \def\GG{\mathbb{G}} \def\HH{\mathbb{H}} \def\II{\mathbb{I}} \def\JJ{\mathbb{J}} \def\KK{\mathbb{K}} \def\LL{\mathbb{L}} \def\MM{\mathbb{M}} \def\NN{\mathbb{N}} \def\OO{\mathbb{O}} \def\PP{\mathbb{P}} \def\QQ{\mathbb{Q}} \def\RR{\mathbb{R}} \def\SS{\mathbb{S}} \def\TT{\mathbb{T}} \def\UU{\mathbb{U}} \def\VV{\mathbb{V}} \def\WW{\mathbb{W}} \def\XX{\mathbb{X}} \def\YY{\mathbb{Y}} \def\ZZ{\mathbb{Z}}  

\def\fa{\mathfrak{a}} \def\fb{\mathfrak{b}} \def\fc{\mathfrak{c}} \def\fd{\mathfrak{d}} \def\fe{\mathfrak{e}} \def\ff{\mathfrak{f}} \def\fg{\mathfrak{g}} \def\fh{\mathfrak{h}} \def\fj{\mathfrak{j}} \def\fk{\mathfrak{k}} \def\fl{\mathfrak{l}} \def\fm{\mathfrak{m}} \def\fn{\mathfrak{n}} \def\fo{\mathfrak{o}} \def\fp{\mathfrak{p}} \def\fq{\mathfrak{q}} \def\fr{\mathfrak{r}} \def\fs{\mathfrak{s}} \def\ft{\mathfrak{t}} \def\fu{\mathfrak{u}} \def\fv{\mathfrak{v}} \def\fw{\mathfrak{w}} \def\fx{\mathfrak{x}} \def\fy{\mathfrak{y}} \def\fz{\mathfrak{z}}
\def\fgl{\mathfrak{gl}}  \def\fsl{\mathfrak{sl}}  \def\fso{\mathfrak{so}}  \def\fsp{\mathfrak{sp}}  
\def\GL{\mathrm{GL}} \def\SL{\mathrm{SL}}  \def\SP{\mathrm{SL}}

\def\<{\langle} \def\>{\rangle}
\def\({\<\!\<}\def\){\>\!\>}
\def\ad{\mathrm{ad}} 
\def\Aut{\mathrm{Aut}}
\def\dim{\mathrm{dim}} 
\def\End{\mathrm{End}} 
\def\ev{\mathrm{ev}} 
\def\half{\hbox{$\frac12$}}
\def\img{\mathrm{img}}
\def\Ind{\mathrm{Ind}}
\def\Hom{\mathrm{Hom}} 
\def\Fn{\mathrm{Fn}} 
\def\Fr{\mathcal{F}\mathrm{r}}
\def\hgt{\mathrm{ht}} 
\def\id{\mathrm{id}} 
\def\qtr{\mathrm{qtr}} 
\def\sgn{\mathrm{sgn}}
\def\supp{\mathrm{supp}}
\def\tr{\mathrm{tr}} 
\def\Tor{\mathrm{Tor}} 
\def\vep{\varepsilon}
\def\f{\varphi}

\usepackage{mathabx}
\def\acts{\lefttorightarrow}



\def\Obj{\mathrm{Obj}}
\def\normeq{\unlhd}
\def\Set{{\cS\mathrm{et}}}
\def\Fin{{\cF\mathrm{inSet}}}
\def\Set{{\cS\mathrm{et}}}
\def\Grp{{\cG\mathrm{rp}}}
\def\Ab{{\cA\mathrm{b}}}
\def\Mod{{\cM\mathrm{od}}}
\def\ab{\mathrm{ab}}

\def\Fix{\mathrm{Fix}}
 \def\triv{\mathrm{triv}}
 
% Arrows:
\newcommand\xdhrightarrow[2][]{%
  \mathrel{\ooalign{$\xrightarrow[#1\mkern4mu]{#2\mkern4mu}$\cr%
  \hidewidth$\rightarrow\mkern4mu$}}
}
%\newcommand\dhrightarrow{%
%  \mathrel{\ooalign{$\rightarrow$\cr%
%  $\mkern3.5mu\rightarrow$}}
%}
\def\dhrightarrow{\twoheadrightarrow}
\def\dhleftarrow{\twoheadleftarrow}


%%%%%%%%%%%%%%%%%%%%%%%%%%%%%% 
%%%%%%%%%%%%%%%%%%%%%%%%%%%%%%

\def\HW{10}
\def\DUE{5/14/2021}

\title[Homework \HW]{Homework \HW \\
Math B4900\\
\small Due: \DUE}
\author{}
%\date{}                                           % Activate to display a given date or no date

\begin{document}
%\maketitle %%% COMMENT THIS OUT and UNCOMMENT the following to give yourself a good assignment header:
\begin{flushright}
Chris Hayduk\\
Math B4900\\
Homework \HW\\
\DUE
\end{flushright}


Let $A$ be a ring with 1.
\begin{enumerate}[I ]
\item \textbf{Lie algebras.}
\begin{enumerate}[1.]
 \item For $x, y \in M_n(\CC)$, show that the binary operation $[x,y] = xy - yx$ is bilinear, skew symmetric, and satisfies the Jacobi identity. 
 
 \begin{proof}
 We have,
 \begin{align*}
 [rx,y] &= (rx)y - y(rx)\\
 &= r(xy) - r(yx)\\
 &= r[x,y]
 \end{align*}
 
 and,
 \begin{align*}
 [x,y \cdot s] &= x(y \cdot s) - (y \cdot s)x\\
 &= (xy) \cdot s - (yx) \cdot s\\
 &= [x,y] \cdot s
 \end{align*}
 
 Thus, the operator is bilinear. Now note that,
 \begin{align*}
 [x, y] &= xy - yx\\
 &= - (yx - xy)\\
 &= [y, x]
 \end{align*}
 
 Hence, the operator is skew symmetric. Lastly, for $x, y, z \in M_n(\CC)$,
 \begin{align*}
 [x, [y,z]] + [y, [z,x]] + [z, [x,y]] &= [x, yz - zy] + [y, zx - xz] + [z, xy - yx]\\
 &= x(yz-zy) - (yz-zy)x + y(zx-xz) - (zx-xz)y\\
 &+ z(xy-yx) - (xy-yx)z\\
 &= xyz - xzy - yzx + zyx + yzx - yxz - zxy + xzy + zxy - zyx - xyz + yxz\\
 &= 0
 \end{align*}
 
 Thus the operator satisfies the Jacobi identity.
 \end{proof}

\item Show that $\fsl_n(\CC) \subseteq M_n(\CC)$ is, indeed, a closed Lie algebra. \\
{[Namely, that if $\tr(x) = \tr(y) = 0$, then $\tr(xy-yx) = 0$ and $\tr(x + \alpha y) = 0$ for all $\alpha \in \CC$.]}

\begin{proof}
Let $x, y \in \fsl_n(\CC)$. Note that $\tr(xy) = \sum_{i,j} x_{ij} y_{ij}$ and $\tr(xy) = tr(yx)$. We have that, for fixed $i, j$, that $(xy)_{ij} = \sum_k = x_{ik}y_{kj}$. Lastly, we also have that $\tr(xy - yx) = \tr(xy) - \tr(yx)$. Thus, using all of the above formulas, we have,
\begin{align*}
\tr(xy - yx) &= \tr(xy) - \tr(yx)\\
&= \tr(xy) - \tr(xy)\\
&= 0
\end{align*}
\end{proof}

\item Let $L(d)$ be the simple $\fsl_2(\CC)$-module with dimension $d+1$. Show that for all $d \ge 1$, 
$$L(d) \otimes L(1) \cong L(d-1) \oplus L(d+1).$$
{[Recall that we have a canonical action of a Lie algebra $\fg$ on the tensor product of two of its modules (see Hopf algebras).]}

\end{enumerate}
 
\item 
\textbf{Characters.} Let $G$ be a finite group, and let $A = \CC G$. 
\begin{enumerate}[1.]
\item \textbf{Characters and tensor products.} Let 
$$\rho: A \to \End(U) \quad \text{ and } \psi: A \to \End(V)$$
be finite-dimensional representations of $A$ (so that $U$ and $V$ are $A$-modules). Let $\cB = \{e_1, \dots, e_m\}$ and $\cB' =\{f_1, \dots, f_n\}$ be ordered bases of $U$ and $V$, respectively. 

For $g \in G$, suppose 
$$\rho(g) = \sum_{i,j=1}^m \alpha_{i,j}E_{i,j}  = 
	\begin{pmatrix} 
			\alpha_{1,1} & \cdots & \alpha_{1,m} \\ 
			\vdots & \ddots& \vdots \\
			\alpha_{m,1} & \cdots & \alpha_{m,m} \end{pmatrix} \quad \text{and} \quad 
	\psi(g) = \sum_{i,j=1}^m \beta_{i,j}E_{i,j}  = 
	\begin{pmatrix} 
			\beta_{1,1} & \cdots & \beta_{1,n} \\ 
			\vdots & \ddots& \vdots \\
			\beta_{n,1} & \cdots & \beta_{n,n} \end{pmatrix},$$
with respect to the bases $\cB$ and $\cB'$, respectively. Namely, 
$$g \cdot e_i = \sum_{\ell=1}^m \alpha_{\ell,i}e_\ell \quad \text{and} \quad g \cdot f_i = \sum_{\ell=1}^n \beta_{\ell,i}f_\ell,$$
for all $i$. Recall that $\cB \times \cB' = \{e_i \otimes f_j\}$ forms a basis of $U \otimes V$, and put the lexicographic order on it (i.e.\ $\cB \times \cB' = \{e_1 \otimes f_1, e_1 \otimes f_1, \dots, e_1\otimes f_n, e_2 \otimes f_1, \dots, e_m \otimes f_n\}$).
\begin{enumerate}[(a)]
\item Compute the action of $g$ on each $e_i \otimes f_j$ for each $i,j$; and give the matrix for $(\rho \otimes \psi)(g)$, with respect to the ordered basis $\cB \times \cB'$, where $\rho \otimes \psi$ is the representation associated to the canonical action of $FG$ on $U \otimes V$.

\begin{proof}
Fix $i \in \{1, \ldots, m\}$ and $j \in \{1, \ldots, n\}$. Then we have,
\begin{align*}
g \cdot e_i \otimes f_j &= (g \cdot e_i) \otimes (g \cdot f_j)\\
&= \left(\sum_{\ell=1}^m \alpha_{\ell,i}e_\ell \right) \otimes \left( \sum_{\ell=1}^n \beta_{\ell,i}f_\ell \right)
\end{align*}

In addition, the matrix for $(\rho \otimes \psi)(g)$ is given by,
\begin{align*}
(\rho \otimes \psi)(g) &= \rho(g) \otimes \psi(g)\\
&= \begin{pmatrix} 
			\alpha_{1,1} & \cdots & \alpha_{1,m} \\ 
			\vdots & \ddots& \vdots \\
			\alpha_{m,1} & \cdots & \alpha_{m,m} \end{pmatrix} \otimes \begin{pmatrix} 
			\beta_{1,1} & \cdots & \beta_{1,n} \\ 
			\vdots & \ddots& \vdots \\
			\beta_{n,1} & \cdots & \beta_{n,n} \end{pmatrix}\\
&= \begin{pmatrix} 
			\alpha_{1,1} \psi(g) & \cdots & \alpha_{1,m} \psi(g) \\ 
			\vdots & \ddots& \vdots \\
			\alpha_{m,1} \psi(g) & \cdots & \alpha_{m,m} \psi(g) \end{pmatrix}\\
&= \begin{pmatrix} 
			\alpha_{1,1} \beta_{1,1} & \alpha_{1,1} \beta_{1,2} & \cdots & \alpha_{1,1} \beta_{1,n} & \alpha_{1,2} \beta_{1,1} & \cdots & \alpha_{1,m} \beta_{1,n} \\ 
			\vdots & \ddots& & & & & \vdots \\
			\alpha_{m,1} \beta_{1,1} & \alpha_{m,1} \beta_{1,2} & \cdots & \alpha_{m,1} \beta_{1,n} & \alpha_{m,2} \beta_{1,1} & \cdots & \alpha_{m,m} \beta_{1,n}\\
			\vdots & & & & & & \vdots \\
			\alpha_{m,1} \beta{n, 1} & \alpha_{m,1} \beta_{n,2} & \cdots & \alpha_{m,1} \beta_{n,n} & \alpha_{m,2} \beta_{n,1} & \cdots & \alpha_{m,m} \beta_{n,n} \end{pmatrix}
\end{align*}

So our new matrix has dimension $mn$.
\end{proof}

\item If 
\begin{enumerate}[\quad]
\item $\chi_\rho$ is the character associated to $\rho$, 
\item $\chi_\psi$ is the character associated to $\psi$, and
\item $\chi_{\rho\otimes\psi}$ is the character associated to $\rho\otimes \psi$, 
\end{enumerate}
use the previous part to prove that $\chi_{\rho\otimes\psi} = \chi_\rho \chi_\psi$.

\begin{proof}
Fix $g \in G$. Then,
\begin{align*}
\chi_\rho(g) &= \tr(\rho(g))\\
&= \sum_{i = 1}^m \alpha_{ii}
\end{align*}

and,
\begin{align*}
\chi_\psi(g) &= \tr(\psi(g))\\
&= \sum_{i=1}^n \beta_{ii}
\end{align*}

Thus, we have,
\begin{align*}
\chi_\rho(g) \chi_\psi(g) &= (\chi_{\rho} \chi_{\psi})(g)\\
&= \sum_{i = 1}^m \alpha_{ii} \cdot \sum_{j = 1}^n\beta_{jj}\\
&= \sum_{i=1}^m \sum_{j=1}^n = \alpha_{ii} \beta_{jj}
\end{align*}

Lastly, using the matrix derived in part (a), we have,
\begin{align*}
\chi_{\rho\otimes\psi}(g) &= \tr( \alpha_{1,1} \psi(g) + \alpha_{2, 2} \psi(g) + \cdots + \alpha_{m,m} \psi(g))\\
&= \tr( \alpha_{1,1} \psi(g)) + \tr(\alpha_{2, 2} \psi(g)) + \cdots + \tr(\alpha_{m,m} \psi(g))\\
&= \alpha_{1,1} \tr(\psi(g)) + \alpha_{2,2} \tr(\psi(g)) + \cdots + \alpha_{m,m} \tr(\psi(g))\\
&= \alpha_{1,1} \sum_{i=1}^n \beta_{ii} + \alpha_{2,2} \sum_{i=1}^n \beta_{ii} + \cdots + \alpha_{m,m} \sum_{i=1}^n \beta_{ii}\\
&= \sum_{i=1}^m \sum_{j=1}^n = \alpha_{ii} \beta_{jj}\\
&= \chi_\rho(g) \chi_\psi(g)
\end{align*}

as required.
\end{proof}

\item Conclude that the set of characters of $\CC G$ forms a subring of $\mathrm{Cl}(\CC G)$. Does this ring have an identity?

\begin{proof}
Since $\rho$, $\psi$ were arbitrary, we have that the set of characters of $\CC G$ is closed under multiplication.
\end{proof}

\end{enumerate}

\item Let $\rho: \CC G \to \End(U)$ be a finite-dimensional representation with character $\chi$. Let $\rho_1, \dots, \rho_r$ be the distinct simple representations of $\CC G$, $V_i$ be the corresponding simple modules, $z_i$ be the corresponding primitive central idempotents, and let $\chi_i$ be their characters. Let $U_i = z_i U  \cong V_i^{\oplus m_i}$ be the isotypic component of $U$ corresponding to $V_i$. Show that $\dim(U_i) = \<\chi,\chi_i\>$. \\
{[\emph{Hint:} This should be a very short jump from things we proved in class.]}




\item \textbf{Burnside's Lemma.}\label{permutation action} Let $G$ act on a set $\Omega$, and let $\Omega_1, \dots, \Omega_\ell$ be the orbits of $\Omega$. Define $U = \CC \Omega$ (the vector space with basis $\Omega$) and extend the action $G \acts \Omega$ linearly to an action $\CC G \acts U$. Let $\rho: \CC G \to \End(U)$ be the associated representation, and $\chi$ be the associated character. 

Fix $g \in G$. Define 
$$\Fix(g) = \{ x \in \Omega ~|~ g \cdot x = x\} $$
to be the number of fixed points under the action of $g$ on $\Omega$.  {[For example, if $G = S_4$ acts on naturally $\Omega = \{1,2,3, 4\}$, and $g = (12)$, then $\{ x \in \Omega ~|~ (12) \cdot x = x\} = \{3, 4\}$. See below for more examples.]} 
\begin{enumerate}[(a)]
\item Argue that $\chi(g) = |\Fix(g)|$.
\begin{proof}
Let $\rho$ be the representation of the action on $U$ with respect to its basis $\omega$. Then $\rho$ is a permutation matrix. Let $e_i, e_j \in \Omega$. Then the $ij$-th entry in $\rho$ is nonzero iff $e_j=g \cdot e_i$. Thus, the nonzero diagonal elements correspond to the fixed points of $g$ on $\Omega$ because, if a diagonal element $i, i$ is nonzero, then we have $e_i = g \cdot e_i$ as required. Thus, the trace of $\rho$ will be equal to the number of fixed points in $g$. But $\chi$ is exactly $\tr(\rho)$, and so we have $\chi(g) = |\Fix(g)|$ as required.
\end{proof}
\item Let $U_i = \CC \Omega_i$. Argue briefly that $U \cong U_1 \oplus \cdots \oplus U_\ell$. For each $i$, let $v_i = \sum_{x \in \Omega_i} x$; show that if $gu = u$ for all $g \in G$ then $u \in \CC\{v_1, \dots, v_\ell\}$ (i.e.\ $\CC\{v_1, \dots, v_\ell\}$ is the isotypic component of $U$ corresponding to the trivial module). 
{[\emph{Hint:} For the second statement, it suffices to look at one $U_i$ at a time: show that for $u_i \in U_i$,  if $g \cdot u_i = u_i$, then $u_i = \alpha v_i$. To do this, use the fact the $G$ acts transitively on $\Omega_i$.]}
\begin{proof}
We have that,
\begin{align*}
U_1 \oplus \cdots \oplus U_{\ell} &= \CC \Omega_1 \oplus \cdots \oplus \CC \Omega_{\ell}\\
&= \CC \Omega_1 \times \cdots \times \CC \Omega_{\ell}\\
&= \CC (\Omega_1 \times \cdots \times \Omega_{\ell})\\
&= \CC \Omega
\end{align*}

Thus, we have that $U \cong U_1 \oplus \cdots U_{\ell}$.\\

Now fix $i$ and let $u_i \in U_i$. Suppose $g \cdot u_i = u_i$.
\end{proof}
\item Prove that $\ell|G| = \sum_{g \in G} |\Fix(G)|$.\\
{[This is the statement that's called \emph{Burnside's Lemma}, though it is due to Frobenius. \emph{Hint:} Compute $\<\chi, \triv\>$, where $\triv$ is the character corresponding to the trivial representation (i.e.\ $\triv: g \mapsto 1$ for all $g \in G$).]} 
\begin{proof}

\end{proof}
\item Compare/contrast Burnside's Lemma to the Orbit-Stabilizer Theorem.
\begin{ans}
Burnside's Lemma is the complement to the Orbit Stabilizer Theorem.
\end{ans}
\end{enumerate}
\end{enumerate}
\end{enumerate}

\vfill
\pagebreak

Examples for \ref{permutation action}: 
\begin{enumerate}
\item The group $S_n$ acts naturally on $\Omega = \{1, \dots, n\}$; the resulting representation is the permutation representation of $S_n$. Specifically, when $n = 3$, the fixed points of each element $g \in S_3$ are given by 
$$\def\arraystretch{1.3}\begin{array}{|c||c|c|c|c|c|c|}\hline
h & 1 & (12) & (23) & (13) & (123) & (132) \\\hline\hline
\Fix(h) &1,2,3 & 3 & 2 & 1 & \text{none}& \text{none}\\\hline
|\Fix(h)| & 3 & 1 & 1 & 1 & 0 & 0\\\hline
\end{array}$$
Here, $S_3$ acts transitively on $\Omega$, so has exactly $\ell = 1$ orbits. And indeed, $3 + 1 + 1 + 1 = 6 = \ell|S_3|$. 

\item Let $Z_6 = \<g\>$ act on $\Omega = \{e_1, \dots,  e_6\}$ by 
$$\TikZ{[xscale=1.2] \foreach \x/\y in {1/0,2/1,3/0}{\node (\x) at (\x,\y) {$\mathbf{e_\x}$};} 
	\draw[->] (1) to [bend left] node[above left]{$g$} (2);
	\draw[->] (2) to [bend left] node[above right]{$g$} (3);
	\draw[->] (3) to [bend left] node[above]{$g$} (1);
	\foreach \x/\y in {4/1.5, 5/2.5}{\node (\x) at (\y,-1.5) {$\mathbf{e_\x}$};} 
	\draw[<->] (4) to node[above]{$g$} (5); 
	\node (6) at (2.5,-3) {$\mathbf{e_6}$}; \draw[->] (6) ..  controls +(-.75,-.5) and +(-.75,.5) .. node[left]{$g$}(6);
	}\ .
\qquad\qquad\text{Then} \quad  \rho(g) = \begin{pmatrix} 0 & 0 & 1 & 0& 0& 0\\1 & 0 & 0& 0& 0& 0\\0&1&0& 0& 0& 0\\0 & 0 & 0 & 0 & 1& 0\\ 0 & 0 & 0 & 1 & 0 & 0 \\ 0 & 0 & 0 & 0 & 0 & 1\end{pmatrix}. \qquad
$$
Here, the orbits are $\Omega_1 = \{e_1, e_2, e_3\}$, $\Omega_2 = \{e_4, e_5\}$, and $\Omega_3 = \{e_6\}$ (since the group is generated by $g$ alone), and 
$$U \cong U_1 \oplus U_2 \oplus U_3, \quad \text{where} \quad U_1=\CC\{e_1, e_2, e_3\},\ U_2=\CC\{e_4, e_5\},\ U_3 = \{e_6\}.$$
And $v_1 = e_1 + e_2 + e_3$, $v_2 = e_4 + e_5$, and $v_3 = e_6$ generate the isotypic component corresponding to the trivial module inside of $U$. 

Further, the fixed points of each element of $Z_6$ are given by 
$$\def\arraystretch{1.3}\begin{array}{|c||c|c|c|c|c|c|}\hline
h & 1 & g & g^2 & g^3 & g^4 & g^5 \\\hline\hline
\Fix(h) & \Omega & e_6 & e_4, e_5, e_6 & e_1, e_2, e_3, e_6 & e_4, e_5, e_6 & e_6\\\hline
|\Fix(h)| & 6 & 1 & 3 &4 & 3 & 1\\\hline
\end{array}$$
Indeed, the number of orbits here is $\ell = 3$, and $6+1+3+4+3+1 = 3*6 = \ell|Z_6|$. 
\end{enumerate}
\end{document}