\documentclass[11pt, reqno]{amsart}
\usepackage[margin=1in]{geometry}    
\geometry{letterpaper}       
%\geometry{landscape}                % Activate for for rotated page geometry
\usepackage[parfill]{parskip}    % Activate to begin paragraphs with an empty line rather than an indent
\usepackage{amsfonts, amscd, amssymb, amsthm, amsmath}
\usepackage{pdfsync}  %leaves makers for tex searching
\usepackage{enumerate}
\usepackage[pdftex,bookmarks]{hyperref}



%%% Theorems %%%--------------------------------------------------------- 
\theoremstyle{plain}
	\newtheorem{thm}{Theorem}[section]
	\newtheorem{lemma}[thm]{Lemma}
	\newtheorem{prop}[thm]{Proposition}
	\newtheorem{cor}[thm]{Corollary}
\theoremstyle{definition}
	\newtheorem*{defn}{Definition}
	\newtheorem{remark}[thm]{Remark}
\theoremstyle{example}
	\newtheorem*{example}{Example}


%%% Environments %%%--------------------------------------------------------- 
\newenvironment{ans}{\medskip \paragraph*{\emph{Answer}.}}{\hfill \break  $~\!\!$ \dotfill \medskip }
\newenvironment{sketch}{\medskip \paragraph*{\emph{Proof sketch}.}}{ \medskip }
\newenvironment{summary}{\medskip \paragraph*{\emph{Summary}.}}{  \hfill \break  \rule{1.5cm}{0.4pt} \medskip }
\newcommand\Ans[1]{\hfill \emph{Answer:} {#1}}


%%% Pictures %%%--------------------------------------------------------- 
%%% If you need to draw pictures, tikzpicture is one good option. Here are some basic things I always use:
%\usepackage{tikz}
%\tikzstyle{V}=[draw, fill =black, circle, inner sep=0pt, minimum size=2pt]
%\newcommand\TikZ[1]{\begin{matrix}\begin{tikzpicture}#1\end{tikzpicture}\end{matrix}}



%%% Color  %%%---------------------------------------------------------
\usepackage{color}
\newcommand{\NOTE}[1]{{\color{blue}#1}}
\newcommand{\MOVED}[1]{{\color{gray}#1}}


%%% Alphabets %%%---------------------------------------------------------
%%% Some shortcuts for my commonly used special alphabets and characters.
\def\cA{\mathcal{A}}\def\cB{\mathcal{B}}\def\cC{\mathcal{C}}\def\cD{\mathcal{D}}\def\cE{\mathcal{E}}\def\cF{\mathcal{F}}\def\cG{\mathcal{G}}\def\cH{\mathcal{H}}\def\cI{\mathcal{I}}\def\cJ{\mathcal{J}}\def\cK{\mathcal{K}}\def\cL{\mathcal{L}}\def\cM{\mathcal{M}}\def\cN{\mathcal{N}}\def\cO{\mathcal{O}}\def\cP{\mathcal{P}}\def\cQ{\mathcal{Q}}\def\cR{\mathcal{R}}\def\cS{\mathcal{S}}\def\cT{\mathcal{T}}\def\cU{\mathcal{U}}\def\cV{\mathcal{V}}\def\cW{\mathcal{W}}\def\cX{\mathcal{X}}\def\cY{\mathcal{Y}}\def\cZ{\mathcal{Z}}

\def\AA{\mathbb{A}} \def\BB{\mathbb{B}} \def\CC{\mathbb{C}} \def\DD{\mathbb{D}} \def\EE{\mathbb{E}} \def\FF{\mathbb{F}} \def\GG{\mathbb{G}} \def\HH{\mathbb{H}} \def\II{\mathbb{I}} \def\JJ{\mathbb{J}} \def\KK{\mathbb{K}} \def\LL{\mathbb{L}} \def\MM{\mathbb{M}} \def\NN{\mathbb{N}} \def\OO{\mathbb{O}} \def\PP{\mathbb{P}} \def\QQ{\mathbb{Q}} \def\RR{\mathbb{R}} \def\SS{\mathbb{S}} \def\TT{\mathbb{T}} \def\UU{\mathbb{U}} \def\VV{\mathbb{V}} \def\WW{\mathbb{W}} \def\XX{\mathbb{X}} \def\YY{\mathbb{Y}} \def\ZZ{\mathbb{Z}}  

\def\fa{\mathfrak{a}} \def\fb{\mathfrak{b}} \def\fc{\mathfrak{c}} \def\fd{\mathfrak{d}} \def\fe{\mathfrak{e}} \def\ff{\mathfrak{f}} \def\fg{\mathfrak{g}} \def\fh{\mathfrak{h}} \def\fj{\mathfrak{j}} \def\fk{\mathfrak{k}} \def\fl{\mathfrak{l}} \def\fm{\mathfrak{m}} \def\fn{\mathfrak{n}} \def\fo{\mathfrak{o}} \def\fp{\mathfrak{p}} \def\fq{\mathfrak{q}} \def\fr{\mathfrak{r}} \def\fs{\mathfrak{s}} \def\ft{\mathfrak{t}} \def\fu{\mathfrak{u}} \def\fv{\mathfrak{v}} \def\fw{\mathfrak{w}} \def\fx{\mathfrak{x}} \def\fy{\mathfrak{y}} \def\fz{\mathfrak{z}}
\def\fgl{\mathfrak{gl}}  \def\fsl{\mathfrak{sl}}  \def\fso{\mathfrak{so}}  \def\fsp{\mathfrak{sp}}  
\def\GL{\mathrm{GL}} \def\SL{\mathrm{SL}}  \def\SP{\mathrm{SL}}

\def\<{\langle} \def\>{\rangle}
\def\ad{\mathrm{ad}} 
\def\Aut{\mathrm{Aut}}
\def\dim{\mathrm{dim}} 
\def\End{\mathrm{End}} 
\def\ev{\mathrm{ev}} 
\def\half{\hbox{$\frac12$}}
\def\Hom{\mathrm{Hom}} 
\def\id{\mathrm{id}} 
\def\qtr{\mathrm{qtr}} 
\def\tr{\mathrm{tr}} 
\def\sgn{\mathrm{sgn}}
\def\vep{\varepsilon}
\def\f{\varphi}


%%%%%%%%%%%%%%%%%%%%%%%%%%%%%% 
%%%%%%%%%%%%%%%%%%%%%%%%%%%%%%

\def\HW{1}
\def\DUE{2/12/2021}

\title[Homework \HW]{Homework \HW \\
Math B4900\\
\small Due: \DUE}
\author{}
%\date{}                                           % Activate to display a given date or no date

\begin{document}
%\maketitle %%% COMMENT THIS OUT and UNCOMMENT the following to give yourself a good assignment header:
\begin{flushright}
Chris Hayduk\\
Math B4900\\
Homework \HW\\
\DUE
\end{flushright}


\begin{enumerate}[1.]
\item \textbf{Class sums.}
Let $R$ be a commutative ring with $1$, and let $G$ be a finite group. Before starting, review our work considering groups acting on themselves by conjugation (Section 4.3 in D\&F). In particular, the conjugacy classes of group elements partition the group. For example, in $S_n$, the conjugacy classes are in bijection with cycle type; in $S_3$ in particular, the classes are 
$$\{1\}, \quad \{(12), (13), (23)\}, \quad \text{ and } \quad \{(123), (132)\}.$$
\begin{enumerate}[(a)]
\item In $RG$, a \emph{class sum} corresponding to a conjugacy class 
$$\cK_g = \{h \in G ~|~ h = aga^{-1} \text{ for some }a \in G\} \qquad  \text{is}  \qquad 
	\kappa_g = \sum_{h \in \cK_g} h.$$
 For example, the class sums in $RS_3$ are 
$$\kappa_1 = 1, \quad \kappa_{(12)} = (12) + (13) +  (23), \quad  \text{ and } \quad 
	\kappa_{(123)} = (123) + (132).$$
Compute the class sums in $RD_8$ and $RA_4$. 

\begin{ans}
Recall that $D_8 = \{1, r, r^2, r^3, s, sr, sr^2, sr^3\}$ with $|r| = 4, |s| = 2, s \neq r^i$ for every $i$, and $rs = sr^{-1}$. The equivalency classes of $D_8$ under conjugacy are 
\begin{align*}
\{1\} \quad \{r, r^3\} \quad \{r^2\} \quad \{s, sr^2\} \quad \{sr, sr^3\}
\end{align*}

Hence, the class sums of $RD_8$ are,
\begin{align*}
\kappa_1 = 1, \quad \kappa_r = r + r^3 \quad \kappa_{r^2} = r^2 \quad \kappa_s = s + sr^2 \quad \kappa_{sr} = sr + sr^3
\end{align*}

Now note that $A_4 = $
\end{ans}

\item For each of the class sums $\kappa$ in $RD_8$, compute $r\kappa r^{-1}$ and $s \kappa s^{-1}$. Use your results to argue that $g \kappa = \kappa g$ for all $g \in D_8$. 

\begin{ans}
We have the following for $r \kappa r^{-1}$,
\begin{align*}
r\kappa_1r^{-1} &= r1r^{-1} = 1\\
r\kappa_rr^{-1} &= r(r + r^3)r^{-1} = (r^2 + r^4)r^{-1} = r + r^3\\
r\kappa_{r^2}r^{-1} &= r(r^2)r^{-1} = r^2\\
r\kappa_{s}r^{-1} &= r(s + sr^2)r^{-1} = (rs + rsr^2)r^{-1} = (sr^{-1} + sr)r^{-1} = sr^{-2} + s = s + sr^2\\
r\kappa_{sr}r^{-1} &= r(sr + sr^3)r^{-1} = (rsr rsr^3)r^{-1} = (s + sr^2)r^{-1} = sr^{-1} + sr = sr + sr^{3}
\end{align*}

Now for $s \kappa s^{-1}$,
\begin{align*}
s\kappa_1s^{-1} &= s1s^{-1} = 1\\
s\kappa_rs^{-1} &= s(r + r^3)s^{-1} = (sr + sr^3)s^{-1} = r^{-1} + r^{-3} = r + r^3\\
s\kappa_{r^2}s^{-1} &= s(r^2)s^{-1} = r^{-2}ss^{-1} = r^2\\
s\kappa_{s}s^{-1} &= s(s + sr^2)s^{-1} = (s^2 + s^2r^2)s^{-1} = s^{-1} + r^2s^{-1} = s + sr^2\\
s\kappa_{sr}s^{-1} &= s(sr + sr^3)s^{-1} = (s^2 + s^2r^3)s^{-1} = s^{-1} + r^{3}s^{-1} = s + sr^3
\end{align*}

Now we know that $D_8$ is generated by $\{r, s\}$. That is, the elements $r, s$ along with the rules mentioned in part $(a)$ allow us to express any element of $D_8$. Hence, for any $g \in D_8$, we can write $$g = r^is^j$$ with $1 \leq i \leq 8$ and $1 \leq j \leq 2$. Thus, by the previous reasoning and above derivations, for any class sum $\kappa$, we have,
\begin{align*}
g\kappa g^{-1} &= r^is^j\kappa s^{-j}r^{-i}\\
&= r^i \kappa r^{-i}\\
&= \kappa
\end{align*}

Now, multiplying by $g$ on the right side of both sides of the equation yields,
\begin{align*}
&(g\kappa g^{-1})g = \kappa g\\
\iff & g \kappa = \kappa g
\end{align*}

as required.
\end{ans}

\item \textbf{Claim:} the center of the group algebra $RG$ is the $R$-span of the class sums of $G$,
$$Z(RG) = R\{\kappa_g ~|~ g \in G\}  = \{ r_1 \kappa_1 + \cdots r_\ell \kappa_\ell ~|~ r _i \in R\},$$
where $\kappa_1, \dots, \kappa_\ell$ denote the $\ell$ class sums of $G$. 

Let's prove it:
\begin{enumerate}[(i)]
\item For each $g \in G$, show  that for all $h \in G$, we have $h \kappa_g h^{-1} = \kappa_g$. Conclude that $a \kappa_g = \kappa_g a$ for all $a \in RG$ (showing that $\kappa_g \in Z(RG)$). 

\begin{proof}
Let $g \in G$ and suppose $\kappa_g$ is the class sum of $g$. Then $\kappa_g = g_1 + g_2 + \cdots + g_k$ for $g_1, \ldots, g_k \in \cK_g$. Now fix $h \in G$. Then we have,
\begin{align*}
h\kappa_gh^{-1} &= h(g_1 + g_2 + \cdots + g_k)h^{-1}
\end{align*}
\end{proof}

\item Use the previous part to show that $r_1 \kappa_1 + \cdots r_\ell \kappa_\ell \in  Z(RG)$ for all $r_i \in R$ (showing that $R\{\kappa_i ~|~ i = 1, \dots \ell\} \subseteq Z(RG)$). 

\begin{proof}
By part $(i)$, we have that for each $g \in G$, $\kappa_g \in Z(RG)$.
\end{proof}

\item Conversely, show that for $a = \sum_{g \in G} s_g g \in RG$, if $h a h^-1 = a$ for all $h \in G$, then $s_g = s_{g'}$ whenever $g$ is conjugate to $g$ (i.e.\ the coefficients are constant across conjugacy classes). {[Hint: Start one at a time: if $h a h^-1 = a$, then compare both sides to get $s_g = s_{h^{-1}gh}$. Try on your examples in part (b) to get started if you need help.]}

\begin{proof}

\end{proof}

\item Let $a \in RG$. Show that if $ha = ah$ for all $h \in G$, then $ba = ab$ for all $b \in RG$. 

\begin{proof}

\end{proof}

\end{enumerate}
\item Let $F$ be a field with $n! \ne 0$ in $F$.\footnote{As usual, as an element of $F$, $n!$ means $1+ 1+ \cdots 1$ ($n!$ terms).} Show that 
$$e_+ = \sum_{\sigma \in S_n} \sigma \qquad \text{ and } \qquad e_- = \sum_{\sigma \in S_n} \sgn(\sigma) \sigma$$ are essential idempotents in $F S_n$ and are central, and compute the corresponding (pure) idempotents. {[\emph{Hint:} Do $e_+$ first, using the fact that any group acts transitively on itself by left multiplication. For $e_-$, do some small examples first, and modify your proof for $e_+$ appropriately.]}
\end{enumerate}

\item \textbf{Vector spaces.} $U, V$, and $W$ denote vector spaces over a common field $F$; $\f$ and $\psi$ denote linear transformations; $\cA$, $\cB$, and $\cC$ denote bases; $A, B$, and $C$ denote matrices in $M_n(F)$. 
\begin{enumerate}[(a)]
\item Let $\f:V \to V$ be a linear map. An element $v \in V$ satisfying $\f(v) = \lambda v$ for some $\lambda \in F$ is called a \emph{weight vector} of \emph{weight $\lambda$} (otherwise known as an \emph{eigenvector} of \emph{eigenvalue} $\lambda$). More restrictively, element $\lambda \in F$ is called a \emph{weight} or \emph{eigenvalue} of $\f$ if it is the weight of some non-zero weight vector of $\f$. Given a weight of $\f$, the \emph{weight space} of $V$ associated to $\lambda$ is 
$$V_\lambda = \{ v\in V ~|~ \f(v) = \lambda v\}$$
(the set of weight vectors in $V$ of weight $\lambda$). 

\medskip

Show that $V_\lambda$ is a subspace of $V$. 

\begin{proof}
By definition, every $v \in V_{\lambda}$ is also an element of $V$. Hence, we have $V_{\lambda} \subset V$. Now observe that $\lambda 0 = 0$ for any $\lambda \in F$. Hence, $0 \in V_{\lambda}$ and thus $V_\lambda$ is non-empty. Now, by the submodule criterion, we just need to show that $x + ry \in V_{\lambda}$ for all $r \in F$ and for all $x, y \in V_{\lambda}$. Let us start by applying $\varphi$ to this element and using the properties of the linearity of $\varphi$,
\begin{align*}
\varphi(x + ry) &= \varphi(x) + r\varphi(y)\\
&= \lambda x + r \lambda y\\
&= \lambda (x + ry)
\end{align*}

Hence, $x + ry \in V_{\lambda}$ and so $V_{\lambda}$ is a subspace of $V$.
\end{proof}

\item Check briefly that $\f(v) = \lambda v$ is equivalent to $(\f - \lambda\cdot \id)(v) = 0$. 

\begin{proof}
We have that $\f(v) \in \text{Hom}_{F}(V, V)$. Now, $\id(rx + y) = rx + y = r\id(x) + \id(y)$ for all $x, y \in V$ and $r \in F$, so $\id \in \text{Hom}_{F}(V, V)$ as well. Thus, we can apply Proposition 2(2) from Section 10.2 of Dummit and Foote to get,
\begin{align*}
&(\f - \lambda\cdot \id)(v) = 0\\
\iff &\f(v) - \lambda\cdot \id(v) = 0\\
\iff &\lambda v - \lambda v = 0\\
\iff &\lambda v = \lambda v
\end{align*}
\end{proof}

\item Given a weight $\lambda$ of $\f$, the \emph{generalized weight space} associated to $\lambda$ is 
$$V^\lambda = \{ v \in V ~|~ (\f - \lambda \cdot \id)^m(v) = 0 \text{ for some } m \in \ZZ_{>0}\}.\footnote{Here, $\psi^m$ means $\psi\circ \psi \circ \cdots \circ \psi$ ($m$ terms).}$$

\begin{enumerate}[(i)]
\item Let 
$$A = \begin{pmatrix} 2 & 1 & 0 \\ 0 & 2 & 0\\ 0&0 & 3\end{pmatrix} \quad \text{ and } \quad 
	v = \begin{pmatrix}0\\1\\0\end{pmatrix}.$$
	Check that $v \in V^2$ but $v \notin V_2$. 

\begin{ans}
Consider,
\begin{align*}
\f(v) &= Av\\
&= \begin{pmatrix}
1 \\ 2 \\ 0
\end{pmatrix}
\end{align*}

Note that,
\begin{align*}
2 v &= \begin{pmatrix}
0 \\ 2 \\ 0
\end{pmatrix}
\end{align*}

Hence, we have that $\f(v) \neq 2v$, and so $v \not\in V_2$. Now we have,
\begin{align*}
Av - 2v &= \begin{pmatrix}
1 \\ 0 \\ 0
\end{pmatrix}
\end{align*}

And thus,
\begin{align*}
(\f - 2 \cdot \id)(Av - 2v) &= A(Av - 2v) - 2(Av - 2v)\\
&= \begin{pmatrix}
2 \\ 0 \\ 0
\end{pmatrix} - \begin{pmatrix}
2 \\ 0 \\ 0
\end{pmatrix}\\
&= \begin{pmatrix}
0 \\ 0 \\ 0
\end{pmatrix} = 0
\end{align*}

As a result, we have that $(\f - 2 \cdot \id)^2(v) = 0$, and so $v \in V^2$

\end{ans}	
	
\item Briefly argue that $V_\lambda \subseteq V^\lambda$.
\begin{ans}
Suppose $v \in V_{\lambda}$. Then $\f(v) = \lambda v$ and, by part (b), we have that $$(\f - \lambda \cdot \id)(v) = (\f - \lambda \cdot \id)^1(v) = 0$$. Hence, $v$ also satisfies the definition of $V^{\lambda}$ with $m = 1$, so $v \in V^{\lambda}$. Since $v$ was arbitrary, this holds for every $v \in V_{\lambda}$ and so $V_{\lambda} \subset V^{\lambda}$.
\end{ans}
\item Show that $V^\lambda$ is a subspace of $V$. \\
	{[\emph{Hint:} If $(\f - \lambda \cdot \id)^m (v) = 0$, then $(\f - \lambda \cdot \id)^n v = 0$ for all integers $n \ge m$.]}


\begin{proof}
By definition, every $v \in V^{\lambda}$ is also an element of $V$. Hence, we have $V^{\lambda} \subset V$. Now observe that, for any $\lambda \in F$,
\begin{align*}
&\f(v) = \lambda 0 = 0v = 0\\
\iff &(\f - \lambda \cdot \id)(0) = 0
\end{align*} 

Hence, $0 \in V^{\lambda}$ and thus $V^\lambda$ is non-empty. Now, by the submodule criterion, we just need to show that $x + ry \in V^{\lambda}$ for all $r \in F$ and for all $x, y \in V^{\lambda}$. Since, $x, y \in V^{\lambda}$, we have that 
\begin{align*}
(\f - \lambda \cdot \id)^{\ell}(x) &= 0\\
(\f - \lambda \cdot \id)^{m}(y) &= 0
\end{align*}

Let $k = \max\{\ell, m\}$. Then by the fact that if $(\f - \lambda \cdot \id)^m (v) = 0$, then $(\f - \lambda \cdot \id)^n v = 0$ for all integers $n \ge m$ and by the fact that linear combinations and compositions of linear functions are linear, we have that,
\begin{align*}
(\f - \lambda \cdot \id)^k (x + ry) &= (\f - \lambda \cdot \id)^k(x) + r (\f - \lambda \cdot \id)^k(y)\\
&= 0 + r0\\
&= 0
\end{align*}

Hence, $x + ry \in V^{\lambda}$ and so $V^{\lambda}$ is a subspace of $V$.
\end{proof}

\end{enumerate}



\end{enumerate}
\end{enumerate}

\vfill


\hrule
\emph{\small To receive credit for this assignment, include the following in your solutions [edited appropriately]:}

\smallskip

\textbf{Academic integrity statement:} I \emph{did not violate} the CUNY Academic Integrity Policy in completing this assignment. \hfill \emph{Christopher Hayduk}


\medskip

\NOTE{For example:\\
\textbf{Academic integrity statement:} I \emph{did not violate} the CUNY Academic Integrity Policy in completing this assignment. \hfill \emph{Zajj B. Daugherty}}
\medskip
\hrule

\vfill


\end{document}